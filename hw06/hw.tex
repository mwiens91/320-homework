% Set up the document
\documentclass{article}

% Page size
\usepackage[
    letterpaper,]{geometry}

% Lines between paragraphs
\setlength{\parskip}{\baselineskip}
\setlength{\parindent}{0pt}

% Math
\usepackage{mathtools}
\usepackage{amssymb}
\usepackage{amsthm}
\usepackage{commath}

% Operators
\DeclareMathOperator{\sgn}{sgn}

% Number sets
\newcommand{\C}{\mathcal{C}}
\newcommand{\N}{\mathbb{N}}
\newcommand{\Q}{\mathbb{Q}}
\newcommand{\R}{\mathbb{R}}
\newcommand{\Z}{\mathbb{Z}}

% Links
\usepackage{hyperref}

% Page numbers at top right
\usepackage{fancyhdr}
\pagestyle{fancy}
\fancyhf{}
\fancyhead[R]{\thepage}
\renewcommand\headrulewidth{0pt}

\begin{document}

\textbf{MATH 320 Homework 6} \\
\textbf{Matt Wiens \#301294492} \\
\textbf{2020-03-06}

1. Consider the mapping $f: (\R, |\cdot|) \to (\R, |\cdot|)$ given by
%
\begin{equation*}
   f(x)
    = \begin{cases}
        \sin \frac{1}{x}, & x \neq 0 \\
        \alpha, & x = 0
   \end{cases}
   .
\end{equation*}
%
Use the definition of continuity in topological spaces to show that $f$
cannot be made into a continuous function no matter which $\alpha$ one
chooses.

\begin{proof}

Suppose for contradiction that $f$ was continuous on its domain. Then
for all open sets $W \subseteq \R$, we would have that $f^{-1}(W)$ is
open.

\textbf{Case 1:} Suppose $\alpha \in (-1, 1)$. Let $S = f^{-1} \del{(-1,
1)}$, an open set by assumption. Since $0 \in S$ and $S$ is open, $0$ is
an interior point of $S$ and there exists $\delta > 0$ such that
$(-\delta, \delta) \subseteq S$ and $f\del{(-\delta, \delta)} \subseteq
(-1, 1)$. Take $k$ sufficiently large such that
%
\begin{equation*}
    x_k = \frac{1}{\pi / 2 + 2 \pi k} < \delta
    .
\end{equation*}
%
Then $x_k \in (-\delta, \delta)$ but $f(x_k) = 1$, which contradicts
$f\del{(-\delta, \delta)} \subseteq (-1, 1)$.

\textbf{Case 2:} Suppose $\alpha = -1$. Then follow the same logic as in
case 1, but use the open interval $(-1 - \epsilon, 1)$ instead of $(-1,
1)$ for any $\epsilon > 0$.

\textbf{Case 3:} Suppose $\alpha = 1$. Then let $S = f^{-1}\del{(-1, 1 +
\epsilon)}$ for any $\epsilon > 0$. Using similar logic as in case 1,
there exists a $\delta > 0$ such that $f\del{(-\delta, \delta)}
\subseteq (-1, 1 + \epsilon)$. Take $k$ sufficiently large such that
%
\begin{equation*}
    x_k = - \frac{1}{\pi / 2 + 2 \pi k} > - \delta
    .
\end{equation*}
%
Then $x_k \in (-\delta, \delta)$ but $f(x_k) = -1$, which is a
contradiction.

\textbf{Case 4:} Suppose $|\alpha| > 1$. Then there exists $\epsilon >
0$ such that $N_\epsilon(\alpha) \cup [-1, 1] = \emptyset$. Therefore,
$f^{-1}\del{N_\epsilon(\alpha)} = \cbr{\alpha}$, which is not an open
set. This contradicts the continuity assumption on $f$.

Thus, for all choices of $\alpha$, $f$ cannot be continuous

\end{proof}

\newpage

2. Let $(X, T_X)$ be a topological space. Consider the function $f: X
   \to \R$ where $\R$ is equipped with the Euclidean metric. Show that
   the set $\cbr{x \in X : f(x) = 2}$ must be closed.

\begin{proof}

\end{proof}

\newpage

3. Let $\Q$ be the set of rational numbers in $\R$. Suppose $f: (\Q,
   |\cdot|) \to (\R, |\cdot|)$ is uniformly continuous. Show that there
   exists a continuous function $F: (\R, |\cdot|) \to (\R, |\cdot|)$
   such that $F(x) = f(x)$ for any $x \in \mathbb{Q}$. In other words,
   $f$ can be extended to a continuous function on $\R$.

\begin{proof}

\end{proof}

\newpage

4. Let $d_\N$ be the discrete metric on $\N$. Find a mapping $f: (\Q,
   |\cdot|) \to (\N, d_\N)$ such that $f$ is continuous and surjective.

\textit{Solution.}
hey

\end{document}
