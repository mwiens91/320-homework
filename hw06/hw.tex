% Set up the document
\documentclass{article}

% Page size
\usepackage[
    letterpaper,]{geometry}

% Lines between paragraphs
\setlength{\parskip}{\baselineskip}
\setlength{\parindent}{0pt}

% Math
\usepackage{mathtools}
\usepackage{amssymb}
\usepackage{amsthm}
\usepackage{commath}

% Operators
\DeclareMathOperator{\sgn}{sgn}

% Number sets
\newcommand{\C}{\mathcal{C}}
\newcommand{\N}{\mathbb{N}}
\newcommand{\Q}{\mathbb{Q}}
\newcommand{\R}{\mathbb{R}}
\newcommand{\Z}{\mathbb{Z}}

% Links
\usepackage{hyperref}

% Page numbers at top right
\usepackage{fancyhdr}
\pagestyle{fancy}
\fancyhf{}
\fancyhead[R]{\thepage}
\renewcommand\headrulewidth{0pt}

\begin{document}

\textbf{MATH 320 Homework 6} \\
\textbf{Matt Wiens \#301294492} \\
\textbf{2020-03-06}

1. Consider the mapping $f: (\R, |\cdot|) \to (\R, |\cdot|)$ given by
%
\begin{equation*}
   f(x)
    = \begin{cases}
        \sin \frac{1}{x}, & x \neq 0 \\
        \alpha, & x = 0
   \end{cases}
   .
\end{equation*}
%
Use the definition of continuity in topological spaces to show that $f$
cannot be made into a continuous function no matter which $\alpha$ one
chooses.

\begin{proof}

Suppose for contradiction that $f$ was continuous on its domain. Then
for all open sets $W \subseteq \R$, we would have that $f^{-1}(W)$ is
open.

\textbf{Case 1:} Suppose $\alpha \in (-1, 1)$. Let $S = f^{-1} \del{(-1,
1)}$, an open set by assumption. Since $0 \in S$ and $S$ is open, $0$ is
an interior point of $S$ and there exists $\delta > 0$ such that
$(-\delta, \delta) \subseteq S$ and $f\del{(-\delta, \delta)} \subseteq
(-1, 1)$. Take $k$ sufficiently large such that
%
\begin{equation*}
    x_k = \frac{1}{\pi / 2 + 2 \pi k} < \delta
    .
\end{equation*}
%
Then $x_k \in (-\delta, \delta)$ but $f(x_k) = 1$, which contradicts
$f\del{(-\delta, \delta)} \subseteq (-1, 1)$.

\textbf{Case 2:} Suppose $\alpha = -1$. Then follow the same logic as in
case 1, but use the open interval $(-1 - \epsilon, 1)$ instead of $(-1,
1)$ for any $\epsilon > 0$.

\textbf{Case 3:} Suppose $\alpha = 1$. Then let $S = f^{-1}\del{(-1, 1 +
\epsilon)}$ for any $\epsilon > 0$. Using similar logic as in case 1,
there exists a $\delta > 0$ such that $f\del{(-\delta, \delta)}
\subseteq (-1, 1 + \epsilon)$. Take $k$ sufficiently large such that
%
\begin{equation*}
    x_k = - \frac{1}{\pi / 2 + 2 \pi k} > - \delta
    .
\end{equation*}
%
Then $x_k \in (-\delta, \delta)$ but $f(x_k) = -1$, which is a
contradiction.

\textbf{Case 4:} Suppose $|\alpha| > 1$. Then there exists $\epsilon >
0$ such that $N_\epsilon(\alpha) \cap [-1, 1] = \emptyset$. Therefore,
$f^{-1}\del{N_\epsilon(\alpha)} = \cbr{\alpha}$, which is not an open
set. This contradicts the continuity assumption on $f$.

Thus, for all choices of $\alpha$, $f$ cannot be continuous

\end{proof}

\newpage

2. Let $(X, T_X)$ be a topological space. Consider the function $f: X
   \to \R$ where $\R$ is equipped with the Euclidean metric. Show that
   if $f$ is continuous on $X$ then the set $\cbr{x \in X : f(x) = 2}$
   must be closed.

\begin{proof}

Let $S = \R \setminus \cbr{2}$, an open set. Because $f$ is continuous,
it follows that $f^{-1}(S)$ is open. Therefore
%
\begin{equation*}
    \del{f^{-1}(S)}^c = f^{-1}(\cbr{2})
\end{equation*}
%
is closed.

\end{proof}

\newpage

3. Let $\Q$ be the set of rational numbers in $\R$. Suppose $f: (\Q,
   |\cdot|) \to (\R, |\cdot|)$ is uniformly continuous. Show that there
   exists a continuous function $F: (\R, |\cdot|) \to (\R, |\cdot|)$
   such that $F(x) = f(x)$ for any $x \in \mathbb{Q}$. In other words,
   $f$ can be extended to a continuous function on $\R$.

\begin{proof}

For every $x \in \R \setminus \Q$, let $\cbr{a_{xn}} \subset \Q$ be a
sequence such that $\cbr{a_{xn}} \to x$.

First, we want to show that for each such $x$, $\cbr{f(a_{xn})}$ is
Cauchy (and hence converges in $\R$). Fix any $\epsilon > 0$. Since $f$
is uniformly continuous, there exists $\delta > 0$ such that $|x - y| <
\delta$ implies that $|f(x) - f(y)| < \epsilon$. Since $\cbr{a_{xn}}$ is
convergent, it is Cauchy, and there exists $N > 0$ such that $n, m \geq
N$ guarantee $|a_{nx} - a_{mx}| < \delta$. Therefore, for all $m, n \geq
N$, $|f(a_{nx}) - f(a_{mx})| < \epsilon$.

Now that we've established that $\cbr{f(a_{xn})}$ is convergent, let us
verify that if $\cbr{a_{xn}}, \cbr{b_{xn}} \subset \Q$ are two sequences
that converge to $x$ then $\lim_{n \to \infty} f(a_{xn}) = \lim_{n \to
\infty} f(b_{xn})$. Fix any $\delta > 0$ and let $N_1 > 0$ be such that
for all $n \geq N_1$, $|a_{xn} - x| < \delta / 2$ and $N_2 > 0$ be such
that for all $n \geq N_2$, $|b_{xn} - x| < \delta / 2$. Let $N =
\max\cbr{N_1, N_2}$. Then by the triangle inequality, for all $n \geq
N$,
%
\begin{equation*}
    |a_{xn} - b_{xn}| \leq |a_{xn} - x| + |b_{xn} - x| < \delta
    .
\end{equation*}
%
Since $f$ is uniformly continuous, for any $\epsilon > 0$, there exists
$\delta > 0$ such that $|a_{xn} - b_{xn}| < \delta$ implies $|f(a_{xn})
- f(b_{xn})| < \epsilon$. Combining this with our above result, there thus
exists an $N > 0$ such that for all $n \geq N$ we have $|a_{xn} -
b_{xn}| < \delta$ and also $|f(a_{xn}) - f(b_{xn})| < \epsilon$. Hence
we have shown that $\lim_{n \to \infty} f(a_{xn}) = \lim_{n \to \infty}
f(b_{xn})$.

Finally, define $F: (\R, |\cdot|) \to (\R, |\cdot|)$ by
%
\begin{equation*}
    F(x) =
        \begin{cases}
            \lim_{n \to \infty} f(a_{xn}), & x \in \R \setminus \Q \\
            f(x), & x \in \Q
        \end{cases}
        .
\end{equation*}
%
The above function $F$ is continuous at every point $x \in \R$ since if
$\cbr{a_n}$ is any sequence such that $\cbr{a_n} \to x$, we have that
$F(a_n) \to F(x)$ (this holds by the extension we've developed and the
underlying continuity of $f$ on $\Q$). Since $F$ is continuous at every
point on its domain, it is continuous on its domain.

\end{proof}

\newpage

4. Let $d_\N$ be the discrete metric on $\N$. Find a mapping $f: (\Q,
   |\cdot|) \to (\N, d_\N)$ such that $f$ is continuous and surjective.

\textit{Solution.}
Recall that in any discrete metric space, all subsets of the space are
open. Let $S_n = \del[0]{n - 3 + \sqrt{2}, n - 2 + \sqrt{2}} \subset \Q$
and $T_n = \del[0]{-n + 2 - \sqrt{2}, -n + 3 - \sqrt{2}} \subset \Q$ for
each $n \in \Z$. Let $f$ be the mapping which takes any $x \in \Q$ and
returns $n \in \N$ such that $|x| \in S_n$. The domain of $f$ is clearly
$\Q$, since the endpoints of the $S_n$ intervals are not elements of
$\Q$ and hence for each $x \in \Q$, $|x| \in S_n$ for some $n$. Now,
$f$ is surjective, since for any $n \in N$, $f^{-1}(n) = S_n \cup T_n
\neq \emptyset$. Also, noting that $\cbr{n} \in \N$ is open, the set
$f^{-1}(n) = S_n \cup T_n$ is also open (since the endpoints of the
$S_n$ intervals $n + \sqrt{2} \not\in \Q$ and the $T_n$ intervals $-n -
\sqrt{2} \not\in \Q$ for all $n \in \Z$). Hence, for any open set
$\Gamma \subseteq \N$, $f^{-1}(\Gamma) = \cup_{n \in \Gamma} S_n \cup
T_n$ is open. Therefore $f$ is also continuous.

\end{document}
