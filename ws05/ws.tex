% Set up the document
\documentclass{article}

% Page size
\usepackage[
    letterpaper,]{geometry}

% Lines between paragraphs
\setlength{\parskip}{\baselineskip}
\setlength{\parindent}{0pt}

% Math
\usepackage{mathtools}
\usepackage{amssymb}
\usepackage{amsthm}
\usepackage{commath}

% Operators
\DeclareMathOperator{\diam}{diam}

% For closure bar
\newcommand*\clos[1]{\mkern 1.5mu\overline{\mkern-1.5mu#1\mkern-1.5mu}\mkern 1.5mu}

% Number sets
\newcommand{\C}{\mathcal{C}}
\newcommand{\N}{\mathbb{N}}
\newcommand{\Q}{\mathbb{Q}}
\newcommand{\R}{\mathbb{R}}
\newcommand{\Z}{\mathbb{Z}}

% Links
\usepackage{hyperref}

% Page numbers at top right
\usepackage{fancyhdr}
\pagestyle{fancy}
\fancyhf{}
\fancyhead[R]{\thepage}
\renewcommand\headrulewidth{0pt}

\begin{document}

\textbf{MATH 320 Worksheet 5} \\
\textbf{Matt Wiens \#301294492} \\
\textbf{2020-03-05}

1. Let $f$ be a continuous mapping from $(X, T_X)$ to $(Y, T_Y)$. Let $E
   \subseteq X$ be dense. Show that $f(E)$ must be dense in $f(X)$.

\begin{proof}

Suppose for contradiction that $f(E)$ is not dense in $f(X)$. Then there
is an interior point $y \in f(X)$ such that $y \not\in f(E)$. Let $Y$ be
an open neighbourhood of $y$ such that $Y \cap f(E) = \emptyset$.
Because $f$ is continuous, $S = f^{-1}(Y)$ is open. Since we have that
$S \subset X$ is open and $S \cap E = \emptyset$, there exists an
interior point $x \in X$ such that $x \not\in E$, which implies that $E$
is not dense in $X$, which is a contradiction.

\end{proof}

\newpage

2. Suppose $A, B$ are disjoint sets in a metric space $(X, d)$, $A$ is
   compact, and $B$ is closed. Show that
%
\begin{equation*}
    \inf_{x \in A, y \in B} d(x, y) > 0
    .
\end{equation*}
%
Show that this assertion fails if $A$ is only closed.

\begin{proof}

Recall that for any closed set $Y$, the function $d(\, \cdot, Y): A \to
\R$ defined by $d(x, Y) = \inf \cbr{d(x, y): y \in Y}$ is continuous.
Also recall that continuous functions on compact sets achieve their
minimum.

Suppose for contradiction that
%
\begin{equation*}
    \inf_{x \in A, y \in B} d(x, y) = 0
    .
\end{equation*}
%
Then if we let $d(\, \cdot, B): A \to \R$, then our assumption can be
rewritten as
%
\begin{equation*}
    \inf_{x \in A} d(x, B) = 0
    .
\end{equation*}
%
However, since continuous functions on compact sets achieve their
minimum, there must exist $x \in A$ such that
%
\begin{equation}
    d(x, B) = 0
    \label{eq:2-contra}
    .
\end{equation}
%
This further implies that there exists a limit point $y$ of $B$ such
that~\eqref{eq:2-contra} holds. But since $B$ is closed, we must have
that $y \in B$. But $d(x, y) = 0$ implies that $x \in B$ and $y \in A$,
both of which are contradictions, since $A$ and $B$ are disjoint.

However, if $A$ is only closed, then there is no guarantee that there is
an $x \in A$ such that~\eqref{eq:2-contra} holds. For example, let $A =
\N$, $B = \cbr{n + \frac{1}{n}: n \in \N}$. Both of these sets are
trivially closed, and the smallest distance between any two points in
these sets (in the $|\cdot|$ metric) is $\frac{1}{n}$; hence we clearly
have that
%
\begin{equation*}
    \inf_{x \in A, y \in B} d(x, y) = 0
    .
\end{equation*}

\end{proof}

\newpage

3. Let $(X, T_X)$ and $(Y, T_Y)$ be two topological spaces. Suppose $X$
   is compact and $f: X \to Y$ is continuous and bijective. Show that
   $f^{-1}: Y \to X$ is continuous.

\begin{proof}

Recall that if $f$ is bijective, then $f = \del{f^{-1}}^{-1}$. Also
recall that for continuous functions, images of compact sets are
compact. Finally, recall that closed subsets of compact sets are compact.

Hence what we need to show is that if $A \subset X$ is open, then $f(A)$
is open in $Y$. Let $A \subset X$ be open. Then $A^c \subseteq X$ is
closed, and because $X$ is compact, $A^c$ is also compact. Then $f(A^c)$
is compact, and hence $f(A^c)^c$ is open. Because $f$ is bijective, it
follows that $f(A^c)^c = f(A)$, an open set.

\end{proof}

\end{document}
