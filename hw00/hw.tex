% Set up the document
\documentclass{article}

% Page size
\usepackage[
    letterpaper,]{geometry}

% Lines between paragraphs
\setlength{\parskip}{\baselineskip}
\setlength{\parindent}{0pt}

% Math
\usepackage{mathtools}
\usepackage{amssymb}
\usepackage{commath}

% Number sets
\newcommand{\Z}{\mathbb{Z}}
\newcommand{\Q}{\mathbb{Q}}
\newcommand{\X}{\mathbb{X}}
\newcommand{\C}{\mathcal{C}}
\newcommand{\N}{\mathbb{N}}
\newcommand{\R}{\mathbb{R}}

% Links
\usepackage{hyperref}

% Page numbers at top right
\usepackage{fancyhdr}
\pagestyle{fancy}
\fancyhf{}
\fancyhead[R]{\thepage}
\renewcommand\headrulewidth{0pt}

\begin{document}

\textbf{MATH 320 Homework 0} \\
\textbf{Matt Wiens \#301294492} \\
\textbf{2020-01-10}

1. Suppose $f: E \to F$ is a mapping. Suppose $A, B \subseteq E$. Prove
   or disprove the following:
%
\begin{enumerate}
    \item $f(A \cup B) = f(A) \cup f(B)$;
    \item $f(A \cap B) = f(A) \cap f(B)$;
    \item $f(A \setminus B) = f(A) \setminus f(B)$.
\end{enumerate}
%
For those that do not hold, modify the relation to obtain a correct
expression.

\textbf{Solution}

\newpage

2. Suppose $A \subseteq \R$ is a set bounded from above. Define
%
\begin{equation*}
   -A = \cbr{-x \in A : x \in A}
   .
\end{equation*}
%
Prove that $\inf(-A)$ exists and
%
\begin{equation*}
    \inf(-A) = -\sup A
    .
\end{equation*}

\textbf{Solution}

\newpage


3. Prove that if $\alpha \in \Q$ satisfies that
%
\begin{equation*}
    \alpha^2 > 2, \qquad \alpha > 0
    ,
\end{equation*}
%
then there exists $\delta \in \Q$ positive such that
%
\begin{equation*}
    \del{\alpha - \delta}^2 > 2
    \quad \text{and} \quad \alpha - \delta > 0
    .
\end{equation*}

\textbf{Solution}

\newpage

4. Consider the root(s) of the equation
%
\begin{equation}
    a_n x^n + a_{n-1} x^{n-1} + \cdots + a_1 x + a_0 = 0
    ,
    \label{eq:q4-1}
\end{equation}
%
where $a_0, \ldots, a_n$ are all integers. Define the set of all the
roots of this type of equation as
%
\begin{equation*}
    A = \cbr{x \in \R : \text{$x$ is a root of \eqref{eq:q4-1} for some $a_0, \cdots, a_n \in \Z$}}
\end{equation*}
%
Prove that $A$ is countable.

\textbf{Solution}

\end{document}
