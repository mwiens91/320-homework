% Set up the document
\documentclass{article}

% Page size
\usepackage[
    letterpaper,]{geometry}

% Lines between paragraphs
\setlength{\parskip}{\baselineskip}
\setlength{\parindent}{0pt}

% Math
\usepackage{mathtools}
\usepackage{amssymb}
\usepackage{amsthm}
\usepackage{commath}

% Number sets
\newcommand{\Z}{\mathbb{Z}}
\newcommand{\Q}{\mathbb{Q}}
\newcommand{\X}{\mathbb{X}}
\newcommand{\C}{\mathcal{C}}
\newcommand{\N}{\mathbb{N}}
\newcommand{\R}{\mathbb{R}}

% Links
\usepackage{hyperref}

% Page numbers at top right
\usepackage{fancyhdr}
\pagestyle{fancy}
\fancyhf{}
\fancyhead[R]{\thepage}
\renewcommand\headrulewidth{0pt}

\begin{document}

\textbf{MATH 320 Homework 0} \\
\textbf{Matt Wiens \#301294492} \\
\textbf{2020-01-10}

1. Suppose $f: E \to F$ is a mapping. Suppose $A, B \subseteq E$. Prove
   or disprove the below equalities. For those that do not hold,
   modify the relation to obtain a correct expression.

(a)
%
\begin{equation*}
    f(A \cup B) = f(A) \cup f(B)
    .
\end{equation*}

\begin{proof}

Suppose $y \in f(A \cup B)$. Then there exists $x \in A \cup B$ such
that $f(x) = y$. Since $x \in A \cup B$, either $x \in A$ or $x \in B$.
Thus $y \in f(A)$ or $y \in f(B)$; or equivalently, $y \in f(A) \cup
f(B)$. Since $y$ was arbitrary, we have $f(A \cup B) \subseteq f(A) \cup
f(B)$.

Now suppose $y \in f(A) \cup f(B)$. Then either $y \in f(A)$ or $y \in
f(B)$. Thus there exists $x \in A$ or $x \in B$ such that $f(x) = y$.
Noting that $x \in A \cup B$, this implies that $y \in f(A \cup B)$.
Hence, we have that $f(A \cup B) \supseteq f(A) \cup f(B)$.

Taking both parts of this proof together, we have that $f(A \cup B) =
f(A) \cup f(B)$.

\end{proof}

(b)
%
\begin{equation*}
    f(A \cap B) = f(A) \cap f(B)
    .
\end{equation*}

\textit{Solution.} The above equality is incorrect. Suppose that $A =
[-1, 0]$, $B = [0, 1]$ and
%
\begin{equation*}
    f(x) =
    \begin{cases}
        1, &x \neq 0 \\
        0, &x = 0
    \end{cases}
\end{equation*}
%
Then $f(A) = f(B) = \cbr{0, 1}$ and $f(A \cap B) = \cbr{0}$. Hence
%
\begin{equation*}
    f(A \cap B)
        = \cbr{0}
        \neq \cbr{0, 1}
        = \cbr{0, 1} \cap \cbr{0, 1}
        = f(A) \cap f(B)
    .
\end{equation*}
%
A modification of the equality that is correct would be
%
\begin{equation*}
    f(A \cap B) \subseteq f(A) \cap f(B)
    .
\end{equation*}

\vspace{3mm}

(c)
%
\begin{equation*}
    f(A \setminus B) = f(A) \setminus f(B)
    .
\end{equation*}

\textit{Solution.} The above equality is also incorrect. Using $A, B, f$
as in the counterexample in part (b), we have that
%
\begin{equation*}
    f(A \setminus B)
        = \cbr{1}
        \neq \varnothing
        = \cbr{0, 1} \setminus \cbr{0, 1}
        = f(A) \setminus f(B)
    .
\end{equation*}
%
A modification of the equality that is correct would be
%
\begin{equation*}
    f(A \setminus B) \supseteq f(A) \setminus f(B)
    .
\end{equation*}

\newpage

2. Suppose $A \subseteq \R$ is a set bounded from above. Define
%
\begin{equation*}
   -A = \cbr{-x : x \in A}
   .
\end{equation*}
%
Prove that $\inf(-A)$ exists and
%
\begin{equation*}
    \inf(-A) = -\sup A
    .
\end{equation*}

\begin{proof}

Since $A \subseteq \R$ we know that $\alpha = \sup A$ exists. Now, for
all $x \in A$, $\alpha \geq x$, or, equivalently, $-\alpha \leq -x$.
Thus, for all $y \in -A$, $-\alpha \leq y$. Hence $-\alpha$ is a lower
bound of $-A$. Let $\beta$ be any lower bound of $-A$. Then for all $y
\in -A$, $\beta \leq y$, or, equivalently, $-\beta \geq -y$. This
implies that for all $x \in A$, $-\beta \geq x$, so $-\beta$ is an upper
bound of $A$. Since $\alpha = \sup A$, we must have that $\alpha \leq
-\beta$, or, equivalently, $-\alpha \geq \beta$. Since $\beta$ was an
arbitrary lower bound of $-A$ and for all such $\beta$, $-\alpha \geq
\beta$, we have that $\inf(-A) = -\alpha = - \sup A$.

\end{proof}

\newpage

3. Prove that if $\alpha \in \Q$ satisfies that
%
\begin{equation}
    \alpha^2 > 2, \qquad \alpha > 0
    ,
    \label{eq:q3-1}
\end{equation}
%
then there exists $\delta \in \Q$ positive such that
%
\begin{equation}
    \del{\alpha - \delta}^2 > 2
    \quad \text{and} \quad \alpha - \delta > 0
    .
    \label{eq:q3-2}
\end{equation}

\begin{proof}

Suppose $\alpha \in \Q$ satisfies \eqref{eq:q3-1}. Then we have that
%
\begin{equation*}
    \alpha > \sqrt 2
\end{equation*}
%
Using that $\Q$ is dense in $\R$, there exists $\beta \in \Q$ such that
%
\begin{equation*}
    \alpha > \beta > \sqrt 2
\end{equation*}
%
Let $\delta = \alpha - \beta$. First, we have that
%
\begin{equation*}
    \alpha - \delta = \beta > \sqrt 2 > 0
    .
\end{equation*}
%
We also have that
%
\begin{equation*}
    \beta^2 > 2
    \iff (\alpha - \delta)^2 > 2
    .
\end{equation*}
%
Hence, $\delta$ satisfies \eqref{eq:q3-2}, and since $\delta \in \Q$, we
have completed the proof.

\end{proof}

\newpage

4. Consider the root(s) of the equation
%
\begin{equation}
    a_n x^n + a_{n-1} x^{n-1} + \cdots + a_1 x + a_0 = 0
    ,
    \label{eq:q4-1}
\end{equation}
%
where $a_0, \ldots, a_n$ are all integers. Define the set of all the
roots of this type of equation as
%
\begin{equation*}
    A = \cbr{x \in \R : \text{$x$ is a root of \eqref{eq:q4-1} for some $a_0, \cdots, a_n \in \Z$}}
\end{equation*}
%
Prove that $A$ is countable.

\begin{proof}

Let $P_n$ be the set of all $n$-th degree polynomials of the form shown
in the LHS of \eqref{eq:q4-1}. Further define
%
\begin{equation*}
    P = \bigcup_{n = 1}^{\infty} P_n
    .
\end{equation*}
%
The function $f_n: P_n \to \Z^n$ defined by
%
\begin{equation*}
    f\del{a_n x^n + a_{n-1} x^{n-1} + \cdots + a_1 x + a_0} = \del{a_n, a_{n-1}, \ldots, a_0}
\end{equation*}
%
is clearly bijective, and since $\Z^n$ is countable, it follows that
each $P_n$ is also countable. Since each $P_n$ is countable, it follows
that $P$ as defined above is countable (the union of countably many
countable sets is countable).

Now, define $R: P \to \Z$, where for each $p \in P$, $R(p)$ is the set
of real roots of $p$. By the Fundamental Theorem of Algebra, $R(p)$ is
finite. Then we have that
%
\begin{equation*}
    A = \bigcup_{p \in P} R(p)
    .
\end{equation*}
%
Since $P$ is countable and for each $p \in P$, $R(p)$ is finite, it
follows that $A$ is countable (the union of countably many finite sets
is countable).

\end{proof}

\end{document}
