% Set up the document
\documentclass{article}

% Page size
\usepackage[
    letterpaper,]{geometry}

% Lines between paragraphs
\setlength{\parskip}{\baselineskip}
\setlength{\parindent}{0pt}

% Math
\usepackage{mathtools}
\usepackage{amssymb}
\usepackage{amsthm}
\usepackage{commath}

% Operators
\DeclareMathOperator{\sgn}{sgn}

% Number sets
\newcommand{\C}{\mathcal{C}}
\newcommand{\N}{\mathbb{N}}
\newcommand{\Q}{\mathbb{Q}}
\newcommand{\R}{\mathbb{R}}
\newcommand{\Z}{\mathbb{Z}}

% Links
\usepackage{hyperref}

% Page numbers at top right
\usepackage{fancyhdr}
\pagestyle{fancy}
\fancyhf{}
\fancyhead[R]{\thepage}
\renewcommand\headrulewidth{0pt}

\begin{document}

\textbf{MATH 320 Homework 5} \\
\textbf{Matt Wiens \#301294492} \\
\textbf{2020-02-28}

1. Suppose the radius of convergence of the power series $\sum_{n =
   1}^\infty a_n x^n$ is $R_0 > 0$. Fix a constant $R$ such that $0 < R
   < R_0$. Consider the metric space $(C[-R, R], d_\infty)$, the
   continuous function space with the supremum metric. Show that
   $\sum_{n = 1}^\infty a_n x^n$ converges in $(C[-R, R], d_\infty)$.

\begin{proof}

\end{proof}

\newpage

2. Suppose $a_n \geq 0$ and $\sum_{n = 1}^\infty a_n$ is divergent.

(a) Show that $\sum_{n = 1}^\infty \frac{a_n}{1 + a_n}$ diverges but
$\sum_{n = 1}^\infty \frac{a_n}{1 + n^2 a_n}$ converges.

\begin{proof}

First we will show that $\sum_{n = 1}^\infty b_n$ diverges, where $b_n =
\frac{a_n}{1 + a_n}$. We will consider two (exhaustive) cases. For the
first case, suppose that infinitely many $a_n \geq 1$. Then we have
infinitely many $b_n \geq \frac{1}{2}$ and clearly $\sum_{n =
1}^\infty b_n$ diverges.

For the second case, instead suppose that only finitely many $a_n \geq
1$. Then there exists $N > 0$ such that for all $n \geq N$, $a_n < 1$.
For all $n \geq N$ we have that $b_n \geq \frac{1}{2} a_n$. Suppose for
contradiction that $b_n < \frac{1}{2} a_n$; then
%
\begin{align*}
    &\frac{a_n}{1 + a_n} < \frac{1}{2} a_n \\
    &\implies 1 + a_n > 2 \\
    &\implies a_n > 1
    ,
\end{align*}
%
which contradicts our assumption on $b_n$. Hence we have that for $n
\geq N$, $b_n \geq \frac{1}{2} a_n$, and thus
%
\begin{equation*}
    \sum_{n = N}^\infty b_n \geq \frac{1}{2} \sum_{n = N}^\infty a_n = \infty
\end{equation*}
%
Therefore $\sum_{n = 1}^\infty b_n = \infty$ and the series diverges (in
both cases).

Now we will show that $\sum_{n = 1}^\infty c_n$ converges, where $c_n
= \frac{a_n}{1 + n^2 a_n}$. Noting that for $a_n > 0$,
%
\begin{equation*}
    0 < c_n = \frac{a_n}{1 + n^2 a_n} < \frac{a_n}{n^2 a_n} = \frac{1}{n^2}
    ,
\end{equation*}
%
and trivially for $a_n = 0$, $c_n = 0 < \frac{1}{n^2}$, because $\sum_{n
= 1}^\infty \frac{1}{n^2}$ converges, by the comparison test, $\sum_{n =
1}^\infty c_n$ also converges.

\end{proof}

(b) What can you say about the convergence/divergence of $\sum_{n =
1}^\infty \frac{a_n}{1 + a_n^2}$ and $\sum_{n = 1}^\infty \frac{a_n}{1 +
n a_n}$? Justify your claims by either proving them or showing
counterexamples.

\textit{Solution.}
For the remaining series, either can converge or diverge. Consider the
series $\sum_{n = 1}^\infty \alpha_n$ with $\alpha_n = \frac{a_n}{1 +
a_n^2}$. If $a_n = \frac{1}{n}$, then
%
\begin{equation*}
    \alpha_n = \frac{1}{n + \frac{1}{n}} > \frac{1}{2 n} = \frac{1}{2} a_n
    .
\end{equation*}
%
Since $\sum_{n = 1}^\infty \frac{1}{2} a_n = \frac{1}{2}\sum_{n =
1}^\infty a_n = \infty$, by the comparison test $\sum_{n = 1}^\infty
\alpha_n$ diverges.

Now consider $a_n = n^2$. Then
%
\begin{equation*}
    \alpha_n = \frac{1}{n^2 + \frac{1}{n^2}} < \frac{1}{n^2}
    .
\end{equation*}
%
Since $\sum_{n = 1}^\infty \frac{1}{n^2}$ converges, by the comparison
test $\sum_{n = 1}^\infty \alpha_n$ also converges.

Now consider the series $\sum_{n = 1}^\infty \beta_n$ with $\beta_n =
\frac{a_n}{1 + n a_n}$. If we again set $a_n = \frac{1}{n}$, then
%
\begin{equation*}
    \beta_n = \frac{1}{n + \frac{1}{n}} > \frac{1}{2 n} = \frac{1}{2} a_n
\end{equation*}
%
and by the exact same logic as above, we have that $\sum_{n = 1}^\infty
\beta_n$ diverges.

Now consider $a_n$ defined by
%
\begin{equation*}
    a_n =
    \begin{cases}
        1,& \text{$n$ is square} \\
        0,& \text{otherwise}
    \end{cases}
    .
\end{equation*}
%
Clearly $\sum_{n = 1}^\infty a_n$ diverges, since it contains infinitely
many terms that are $1$. Now,
%
\begin{equation*}
    \beta_n =
    \begin{dcases}
        \frac{1}{1 + n},& \text{$n$ is square} \\
        0,& \text{otherwise}
    \end{dcases}
    .
\end{equation*}
%
Thus we can re-write
%
\begin{equation*}
    \sum_{n = 1}^\infty \beta_n = \sum_{n = 1}^\infty \frac{1}{1 + n^2}
\end{equation*}
%
since all of the non-squared terms vanish. Since
%
\begin{equation*}
    \frac{1}{1 + n^2} < \frac{1}{n^2}
\end{equation*}
%
and $\sum_{n = 1}^\infty \frac{1}{n^2}$ converges, by the comparison
test $\sum_{n = 1}^\infty \beta_n$ also converges.

\newpage

3. (a) Show that for any $n \geq 1$,
%
\begin{equation}
    0 < \frac{1}{n} - \ln \del{1 + \frac{1}{n}} < \frac{1}{n^2}.
    \label{eq:q3-a}
\end{equation}

\begin{proof}

\end{proof}

(b) Prove that for any $n \geq 1$,
%
\begin{equation}
    1 - \frac{1}{2} + \frac{1}{3} - \frac{1}{4} + \cdots + \frac{1}{2 n - 1} - \frac{1}{2 n}
    = \frac{1}{n + 1} + \frac{1}{n + 2} + \cdots + \frac{1}{2 n}.
    \label{eq:q3-b}
\end{equation}

\begin{proof}

\end{proof}

(c) Combine~\eqref{eq:q3-a} and~\eqref{eq:q3-b} to show that
%
\begin{equation*}
   \sum_{n = 1}^\infty  \frac{(-1)^{n - 1}}{n} = \log 2.
\end{equation*}

\begin{proof}

\end{proof}

(d) Recall that the Riemann theorem (stated in class) asserts that a
conditionally convergent series can converge to any prescribed limit by
re-arrangement. Describe how you would make a re-arrangement of
$\sum_{n=1}^\infty \frac{(-1)^{n - 1}}{n}$ such that the re-arranged
series converges to $2$.

\textit{Solution.}
Hey

\end{document}
