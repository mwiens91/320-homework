% Set up the document
\documentclass{article}

% Page size
\usepackage[
    letterpaper,]{geometry}

% Lines between paragraphs
\setlength{\parskip}{\baselineskip}
\setlength{\parindent}{0pt}

% Math
\usepackage{mathtools}
\usepackage{amssymb}
\usepackage{amsthm}
\usepackage{commath}

% Operators
\DeclareMathOperator{\sgn}{sgn}

% Number sets
\newcommand{\C}{\mathcal{C}}
\newcommand{\N}{\mathbb{N}}
\newcommand{\Q}{\mathbb{Q}}
\newcommand{\R}{\mathbb{R}}
\newcommand{\Z}{\mathbb{Z}}

% Links
\usepackage{hyperref}

% Page numbers at top right
\usepackage{fancyhdr}
\pagestyle{fancy}
\fancyhf{}
\fancyhead[R]{\thepage}
\renewcommand\headrulewidth{0pt}

\begin{document}

\textbf{MATH 320 Homework 5} \\
\textbf{Matt Wiens \#301294492} \\
\textbf{2020-02-28}

1. Suppose the radius of convergence of the power series $\sum_{n =
   1}^\infty a_n x^n$ is $R_0 > 0$. Fix a constant $R$ such that $0 < R
   < R_0$. Consider the metric space $(C[-R, R], d_\infty)$, the
   continuous function space with the supremum metric. Show that
   $\sum_{n = 1}^\infty a_n x^n$ converges in $(C[-R, R], d_\infty)$.

\begin{proof}

\end{proof}

\newpage

2. Suppose $a_n \geq 0$ and $\sum_{n = 1}^\infty a_n$ is divergent.

(a) Show that $\sum_{n = 1}^\infty \frac{a_n}{1 + a_n}$ diverges but
$\sum_{n = 1}^\infty \frac{a_n}{1 + n^2 a_n}$ converges.

\begin{proof}

\end{proof}

(b) What can you say about the convergence/divergence of $\sum_{n =
1}^\infty \frac{a_n}{1 + a_n^2}$ and $\sum_{n = 1}^\infty \frac{a_n}{1 +
n a_n}$? Justify your claims by either proving them or showing
counterexamples.

\textit{Solution.}
hey

\newpage

3. (a) Show that for any $n \geq 1$,
%
\begin{equation}
    0 < \frac{1}{n} - \ln \del{1 + \frac{1}{n}} < \frac{1}{n^2}.
    \label{eq:q3-a}
\end{equation}

\begin{proof}

\end{proof}

(b) Prove that for any $n \geq 1$,
%
\begin{equation}
    1 - \frac{1}{2} + \frac{1}{3} - \frac{1}{4} + \cdots + \frac{1}{2 n - 1} - \frac{1}{2 n}
    = \frac{1}{n + 1} + \frac{1}{n + 2} + \cdots + \frac{1}{2 n}.
    \label{eq:q3-b}
\end{equation}

\begin{proof}

\end{proof}

(c) Combine~\eqref{eq:q3-a} and~\eqref{eq:q3-b} to show that
%
\begin{equation*}
   \sum_{n = 1}^\infty  \frac{(-1)^{n - 1}}{n} = \log 2.
\end{equation*}

\begin{proof}

\end{proof}

(d) Recall that the Riemann theorem (stated in class) asserts that a
conditionally convergent series can converge to any prescribed limit by
re-arrangement. Describe how you would make a re-arrangement of
$\sum_{n=1}^\infty \frac{(-1)^{n - 1}}{n}$ such that the re-arranged
series converges to $2$.

\textit{Solution.}
Hey

\end{document}
