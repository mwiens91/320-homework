% Set up the document
\documentclass{article}

% Page size
\usepackage[
    letterpaper,]{geometry}

% Lines between paragraphs
\setlength{\parskip}{\baselineskip}
\setlength{\parindent}{0pt}

% Math
\usepackage{mathtools}
\usepackage{amssymb}
\usepackage{amsthm}
\usepackage{commath}

% Operators
\DeclareMathOperator{\sgn}{sgn}

% Number sets
\newcommand{\C}{\mathcal{C}}
\newcommand{\N}{\mathbb{N}}
\newcommand{\Q}{\mathbb{Q}}
\newcommand{\R}{\mathbb{R}}
\newcommand{\Z}{\mathbb{Z}}

% Links
\usepackage{hyperref}

% Page numbers at top right
\usepackage{fancyhdr}
\pagestyle{fancy}
\fancyhf{}
\fancyhead[R]{\thepage}
\renewcommand\headrulewidth{0pt}

\begin{document}

\textbf{MATH 320 Homework 5} \\
\textbf{Matt Wiens \#301294492} \\
\textbf{2020-02-28}

1. Suppose the radius of convergence of the power series $\sum_{n =
   1}^\infty a_n x^n$ is $R_0 > 0$. Fix a constant $R$ such that $0 < R
   < R_0$. Consider the metric space $(C[-R, R], d_\infty)$, the
   continuous function space with the supremum metric. Show that
   $\sum_{n = 1}^\infty a_n x^n$ converges in $(C[-R, R], d_\infty)$.

\begin{proof}

The proof here will be very similar to the proof of the Weierstrass M-test.
For $n \in \N$, let
%
\begin{equation*}
    f_n(x) = a_n x^n
    ,
\end{equation*}
%
and
%
\begin{equation*}
    M_n = |a_n| R^n
    .
\end{equation*}
%
Note that for each $n$, and $x \in [-R, R]$, $|f_n(x)| \leq M_n$. Also
note that $\sum_{n = 1}^\infty M_n$ is finite, since $R \in (-R_0,
R_0)$. Now, letting
%
\begin{equation*}
    S_k(x) = \sum_{n = 1}^k f_n(x), \qquad S(x) = \sum_{n = 1}^\infty f_n(x),
\end{equation*}
%
we want to show that for any $\epsilon > 0$ there exists $N \in \N$ such
that for all $k \geq N$
%
\begin{equation*}
    \sup_{x \in [-R, R]}|S_k(x) - S(x)| < \epsilon
    .
\end{equation*}
%
Fix any $\epsilon > 0$. Because $\sum_{n = 1}^\infty M_n$ is finite,
there exists an $N \in \N$ such that $\sum_{n = N}^\infty M_n <
\epsilon$ (this is because the series is finite and thus also the terms
$M_n \to 0$). But then for all $k \geq N$, we have
%
\begin{align*}
    \sup_{x \in [-R, R]} |S_k(x) - S(x)|
        &= \sup_{x \in [-R, R]} \envert{\sum_{n = k}^\infty f_n(x)} \\
        &\leq \sup_{x \in [-R, R]} \sum_{n = k}^\infty |f_n(x)| \\
        &\leq \sum_{n = k}^\infty M_n \\
        &\leq \sum_{n = N}^\infty M_n \\
        &< \epsilon
        .
\end{align*}
%
This shows that $S_k \to S$ in $(C[-R, R], d_\infty)$.


\end{proof}

\newpage

2. Suppose $a_n \geq 0$ and $\sum_{n = 1}^\infty a_n$ is divergent.

(a) Show that $\sum_{n = 1}^\infty \frac{a_n}{1 + a_n}$ diverges but
$\sum_{n = 1}^\infty \frac{a_n}{1 + n^2 a_n}$ converges.

\begin{proof}

First we will show that $\sum_{n = 1}^\infty b_n$ diverges, where $b_n =
\frac{a_n}{1 + a_n}$. We will consider two (exhaustive) cases. For the
first case, suppose that infinitely many $a_n \geq 1$. Then we have
infinitely many $b_n \geq \frac{1}{2}$ and clearly $\sum_{n =
1}^\infty b_n$ diverges.

For the second case, instead suppose that only finitely many $a_n \geq
1$. Then there exists $N > 0$ such that for all $n \geq N$, $a_n < 1$.
For all $n \geq N$ we have that $b_n \geq \frac{1}{2} a_n$. Suppose for
contradiction that $b_n < \frac{1}{2} a_n$; then
%
\begin{align*}
    &\frac{a_n}{1 + a_n} < \frac{1}{2} a_n \\
    &\implies 1 + a_n > 2 \\
    &\implies a_n > 1
    ,
\end{align*}
%
which contradicts our assumption on $b_n$. Hence we have that for $n
\geq N$, $b_n \geq \frac{1}{2} a_n$, and thus
%
\begin{equation*}
    \sum_{n = N}^\infty b_n \geq \frac{1}{2} \sum_{n = N}^\infty a_n = \infty
\end{equation*}
%
Therefore $\sum_{n = 1}^\infty b_n = \infty$ and the series diverges (in
both cases).

Now we will show that $\sum_{n = 1}^\infty c_n$ converges, where $c_n
= \frac{a_n}{1 + n^2 a_n}$. Noting that for $a_n > 0$,
%
\begin{equation*}
    0 < c_n = \frac{a_n}{1 + n^2 a_n} < \frac{a_n}{n^2 a_n} = \frac{1}{n^2}
    ,
\end{equation*}
%
and trivially for $a_n = 0$, $c_n = 0 < \frac{1}{n^2}$, because $\sum_{n
= 1}^\infty \frac{1}{n^2}$ converges, by the comparison test, $\sum_{n =
1}^\infty c_n$ also converges.

\end{proof}

(b) What can you say about the convergence/divergence of $\sum_{n =
1}^\infty \frac{a_n}{1 + a_n^2}$ and $\sum_{n = 1}^\infty \frac{a_n}{1 +
n a_n}$? Justify your claims by either proving them or showing
counterexamples.

\textit{Solution.}
For the remaining series, either can converge or diverge. Consider the
series $\sum_{n = 1}^\infty \alpha_n$ with $\alpha_n = \frac{a_n}{1 +
a_n^2}$. If $a_n = \frac{1}{n}$, then
%
\begin{equation*}
    \alpha_n = \frac{1}{n + \frac{1}{n}} > \frac{1}{2 n} = \frac{1}{2} a_n
    .
\end{equation*}
%
Since $\sum_{n = 1}^\infty \frac{1}{2} a_n = \frac{1}{2}\sum_{n =
1}^\infty a_n = \infty$, by the comparison test $\sum_{n = 1}^\infty
\alpha_n$ diverges.

Now consider $a_n = n^2$. Then
%
\begin{equation*}
    \alpha_n = \frac{1}{n^2 + \frac{1}{n^2}} < \frac{1}{n^2}
    .
\end{equation*}
%
Since $\sum_{n = 1}^\infty \frac{1}{n^2}$ converges, by the comparison
test $\sum_{n = 1}^\infty \alpha_n$ also converges.

Now consider the series $\sum_{n = 1}^\infty \beta_n$ with $\beta_n =
\frac{a_n}{1 + n a_n}$. If we again set $a_n = \frac{1}{n}$, then
%
\begin{equation*}
    \beta_n = \frac{1}{n + \frac{1}{n}} > \frac{1}{2 n} = \frac{1}{2} a_n
\end{equation*}
%
and by the exact same logic as above, we have that $\sum_{n = 1}^\infty
\beta_n$ diverges.

Now consider $a_n$ defined by
%
\begin{equation*}
    a_n =
    \begin{cases}
        1,& \text{$n$ is square} \\
        0,& \text{otherwise}
    \end{cases}
    .
\end{equation*}
%
Clearly $\sum_{n = 1}^\infty a_n$ diverges, since it contains infinitely
many terms that are $1$. Now,
%
\begin{equation*}
    \beta_n =
    \begin{dcases}
        \frac{1}{1 + n},& \text{$n$ is square} \\
        0,& \text{otherwise}
    \end{dcases}
    .
\end{equation*}
%
Thus we can re-write
%
\begin{equation*}
    \sum_{n = 1}^\infty \beta_n = \sum_{n = 1}^\infty \frac{1}{1 + n^2}
\end{equation*}
%
since all of the non-squared terms vanish. Since
%
\begin{equation*}
    \frac{1}{1 + n^2} < \frac{1}{n^2}
\end{equation*}
%
and $\sum_{n = 1}^\infty \frac{1}{n^2}$ converges, by the comparison
test $\sum_{n = 1}^\infty \beta_n$ also converges.

\newpage

3. (a) Show that for any $n \geq 1$,
%
\begin{equation}
    0 < \frac{1}{n} - \log \del{1 + \frac{1}{n}} < \frac{1}{n^2}.
    \label{eq:q3-a}
\end{equation}

\begin{proof}

First we will prove that
%
\begin{equation}
    \frac{1}{n + 1} < \log\del{1 + \frac{1}{n}} < \frac{1}{n}
    \label{eq:q3-a-i}
\end{equation}
%
using the calculus identity
%
\begin{equation*}
    \log\del{1 + \frac{1}{n}} = \int_1^{1 + \frac{1}{n}} \frac{1}{x} \dif x
    .
\end{equation*}
%
Now, in the above integrand we clearly have $1 < t < 1 + \frac{1}{n}$;
rearranging this inequality, we have $\frac{n}{n + 1} < \frac{1}{t} < 1$.
Thus we have
%
\begin{align*}
    &
        \int_1^{1 + \frac{1}{n}} \frac{n}{n + 1} \dif t
        < \int_1^{1 + \frac{1}{n}} \frac{1}{t} \dif t
        < \int_1^{1 + \frac{1}{n}} 1 \dif t
        \\
    &\iff
        \frac{n}{n + 1} \cdot \frac{1}{n}
        < \int_1^{1 + \frac{1}{n}} \frac{1}{t} \dif t
        < \frac{1}{n}
        \\
    &\iff
        \frac{1}{n + 1}
        < \log\del{1 + \frac{1}{n}}
        < \frac{1}{n}
        ,
\end{align*}
%
which proves~\eqref{eq:q3-a-i}. We can obtain~\eqref{eq:q3-a}
from~\eqref{eq:q3-a-i} as follows:
%
\begin{align*}
    &\frac{1}{n + 1} < \log\del{1 + \frac{1}{n}} < \frac{1}{n} \\
    &\iff - \frac{1}{n + 1} > - \log\del{1 + \frac{1}{n}} > - \frac{1}{n} \\
    &\iff \frac{1}{n} - \frac{1}{n + 1} > \frac{1}{n} - \log\del{1 + \frac{1}{n}} > 0 \\
    &\iff \frac{1}{n (n + 1)} > \frac{1}{n} - \log\del{1 + \frac{1}{n}} > 0 \\
    &\implies \frac{1}{n^2} > \frac{1}{n} - \log\del{1 + \frac{1}{n}} > 0
    ,
\end{align*}
%
where the last line is precisely~\eqref{eq:q3-a}.

\end{proof}

(b) Prove that for any $n \geq 1$,
%
\begin{equation}
    1 - \frac{1}{2} + \frac{1}{3} - \frac{1}{4} + \cdots + \frac{1}{2 n - 1} - \frac{1}{2 n}
    = \frac{1}{n + 1} + \frac{1}{n + 2} + \cdots + \frac{1}{2 n}.
    \label{eq:q3-b}
\end{equation}

\begin{proof}

We will proceed by induction. Clearly for $n = 1$,
%
\begin{equation*}
    1 - \frac{1}{2} = \frac{1}{1 + 1}
    .
\end{equation*}
%
Suppose the statement holds for $n \leq k - 1$. Then
%
\begin{equation*}
    1 - \frac{1}{2} + \frac{1}{3} - \frac{1}{4} + \cdots + \frac{1}{2 (k - 1) - 1} - \frac{1}{2 (k - 1)}
    = \frac{1}{(k - 1) + 1} + \frac{1}{(k - 1) + 2} + \cdots + \frac{1}{2 (k - 1)}
\end{equation*}
%
and, simplifying,
%
\begin{equation*}
    1 - \frac{1}{2} + \frac{1}{3} - \frac{1}{4} + \cdots + \frac{1}{2 k - 3} - \frac{1}{2 k - 2}
    = \frac{1}{k} + \frac{1}{k + 1} + \cdots + \frac{1}{2 k - 2}
    .
\end{equation*}
%
Adding
%
\begin{equation*}
    \frac{1}{2 k - 1} - \frac{1}{2 k}
\end{equation*}
%
to both sides, we have
%
\begin{equation*}
    \text{LHS}
        = 1 - \frac{1}{2} + \frac{1}{3} - \frac{1}{4} + \cdots + \frac{1}{2 k - 3} - \frac{1}{2 k - 2}
            + \frac{1}{2 k - 1} - \frac{1}{2 k}
\end{equation*}
%
and
%
\begin{align*}
    \text{RHS}
        &= \frac{1}{k} + \frac{1}{k + 1} + \cdots + \frac{1}{2 k - 2} + \frac{1}{2 k - 1} - \frac{1}{2 k} \\
        &= \frac{1}{k + 1} + \cdots + \frac{1}{2 k - 2} + \frac{1}{2 k - 1} + \frac{1}{k} - \frac{1}{2 k} \\
        &= \frac{1}{k + 1} + \cdots + \frac{1}{2 k - 2} + \frac{1}{2 k - 1} + \frac{1}{2 k}
        .
\end{align*}
%
So the statement is also true for $n = k$. This completes the induction.

\end{proof}

(c) Combine~\eqref{eq:q3-a} and~\eqref{eq:q3-b} to show that
%
\begin{equation*}
   \sum_{n = 1}^\infty  \frac{(-1)^{n - 1}}{n} = \log 2.
\end{equation*}

\begin{proof}

First, we note that~\eqref{eq:q3-b} can be expressed as
%
\begin{equation}
    \sum_{n = 1}^{2 k} (-1)^{n - 1} \frac{1}{n}
    = \sum_{n = 1}^{k} \frac{1}{k + n}
    = \sum_{n = k}^{2 k} \frac{1}{n}
    \label{eq:q3-c-i}
    .
\end{equation}
%
Now, fix any $\epsilon > 0$. Let $N_1 \in \N$ be sufficiently large so
that for all $k \geq N_1$,
%
\begin{equation}
    \sum_{n = k}^{2k} \frac{1}{n^2} < \frac{\epsilon}{2}
    \label{eq:q3-c-ii}
    ;
\end{equation}
%
and let $N_2 \in \N$ be sufficiently large so that
%
\begin{equation}
    \log\del{2 + \frac{1}{N_2}} - \log 2 < \frac{\epsilon}{2}
    \label{eq:q3-c-iii}
    .
\end{equation}
%
Adding $\log\del{1 + \frac{1}{n}}$ through in~\eqref{eq:q3-a} we have
%
\begin{equation*}
    \log\del{1 + \frac{1}{n}} < \frac{1}{n} < \frac{1}{n^2} + \log\del{1 + \frac{1}{n}}
    ,
\end{equation*}
%
which can also be expressed as
%
\begin{equation*}
    \log(n + 1) - \log n < \frac{1}{n} < \frac{1}{n^2} + \log(n + 1) - \log n
    .
\end{equation*}
%
Let $k \geq \max\cbr{N_1, N_2}$. Using the above equation, we have that
%
\begin{equation*}
    \sum_{n = k}^{2k} \del{\log(n + 1) - \log n}
    < \sum_{n = k}^{2k} \frac{1}{n}
    < \sum_{n = k}^{2k} \del{\frac{1}{n^2} + \log(n + 1) - \log n}
\end{equation*}
%
which simplifies to
%
\begin{equation*}
    \log(2k + 1) - \log k
    < \sum_{n = k}^{2k} \frac{1}{n}
    < \log(2k + 1) - \log k + \sum_{n = k}^{2k} \frac{1}{n^2}
    .
\end{equation*}
%
Combining logarithms and simplifying, this becomes
%
\begin{equation*}
    \log\del{2 + \frac{1}{k}}
    < \sum_{n = k}^{2k} \frac{1}{n}
    < \log\del{2 + \frac{1}{k}} + \sum_{n = k}^{2k} \frac{1}{n^2}
\end{equation*}
%
or
%
\begin{equation*}
    0
    < \sum_{n = k}^{2k} \frac{1}{n} - \log\del{2 + \frac{1}{k}}
    < \sum_{n = k}^{2k} \frac{1}{n^2}
    .
\end{equation*}
%
Now, using~\eqref{eq:q3-c-i},~\eqref{eq:q3-c-ii}, and~\eqref{eq:q3-c-iii} we have
%
\begin{align*}
    &0
    < \sum_{n = 1}^{2k} (-1)^{n - 1} \frac{1}{n} - \log 2 - \frac{\epsilon}{2}
    < \frac{\epsilon}{2} \\
    &\implies \envert{\sum_{n = 1}^{2k} (-1)^{n - 1} \frac{1}{n} - \log 2} < \epsilon
    .
\end{align*}
%
Since $\epsilon$ was arbitrary, it follows that $\sum_{n = 1}^\infty
\frac{(-1)^{n - 1}}{n} = \log 2$.

\end{proof}

(d) Recall that the Riemann theorem (stated in class) asserts that a
conditionally convergent series can converge to any prescribed limit by
re-arrangement. Describe how you would make a re-arrangement of
$\sum_{n=1}^\infty \frac{(-1)^{n - 1}}{n}$ such that the re-arranged
series converges to $2$.

\textit{Solution.}
Although finding an explicit formula seems to be extremely difficult (I
tried), the general idea for what we'd want to do is to add positive
terms until we reach $2$, and then add chunks of positive and negative
terms that are either $0$ or converge to $0$. Ideally, we would develop
an explicit formula for this.

\end{document}
