% Set up the document
\documentclass{article}

% Page size
\usepackage[
    letterpaper,]{geometry}

% Lines between paragraphs
\setlength{\parskip}{\baselineskip}
\setlength{\parindent}{0pt}

% Math
\usepackage{mathtools}
\usepackage{amssymb}
\usepackage{amsthm}
\usepackage{commath}

% Number sets
\newcommand{\Z}{\mathbb{Z}}
\newcommand{\Q}{\mathbb{Q}}
\newcommand{\X}{\mathbb{X}}
\newcommand{\C}{\mathcal{C}}
\newcommand{\N}{\mathbb{N}}
\newcommand{\R}{\mathbb{R}}
\newcommand{\V}{\mathbb{V}}

% Handy shortcuts
\newcommand*\clos[1]{\mkern 1.5mu\overline{\mkern-1.5mu#1\mkern-1.5mu}\mkern 1.5mu}

% Links
\usepackage{hyperref}

% Page numbers at top right
\usepackage{fancyhdr}
\pagestyle{fancy}
\fancyhf{}
\fancyhead[R]{\thepage}
\renewcommand\headrulewidth{0pt}

\begin{document}

\textbf{MATH 320 Worksheet 1} \\
\textbf{Matt Wiens \#301294492} \\
\textbf{2020-01-15}

1. Let $X = \R^n$. Verify that the following are metrics on $X$:

(a)
%
\begin{equation*}
    d_1(x, y) = \sum_{i = 1}^n \envert{x_i - y_i}
    .
\end{equation*}

\textit{Solution.}
We need to verify three properties:

Fix any $x, y, z \in X$.

1. Clearly $d_1(x, y) \geq 0$. Also, we have
%
\begin{align*}
    &d_1(x, y) = 0 \\
    &\iff x_i = y_i, \quad \forall i = 1, \ldots, n \\
    &\iff x = y
    .
\end{align*}
%
2. Here we need to verify the symmetry of $d_1$:
%
\begin{align*}
    d_1(x, y) &= \sum_{i = 1}^n \envert{x_i - y_i} \\
              &= \sum_{i = 1}^n \envert{y_i - x_i} \\
              &= d_1(y, x)
    .
\end{align*}
%
3. For this property we will use the triangle inequality of $|\cdot|$:
%
\begin{align*}
    d_1(x, y) + d_1(y, z)
        &= \sum_{i = 1}^n \envert{x_i - y_i} + \sum_{i = 1}^n \envert{y_i - z_i} \\
        &= \sum_{i = 1}^n \del[2]{\envert{x_i - y_i} + \envert{y_i - z_i}} \\
        &\geq \sum_{i = 1}^n \envert{x_i - z_i} \\
        &= d_1(x, z)
    .
\end{align*}

\vspace{3mm}

(b)
%
\begin{equation*}
    d_2(x, y) = \sqrt{\sum_{i = 1}^n {\envert{x_i - y_i}}^2}
    .
\end{equation*}

\textit{Solution.}
Properties 1 and 2 are verified similarly to what was shown in part (a).
Here we will only verify property 3.

Fix any $x, y, z \in X$. Then
%
\begin{align*}
    d_2(x, z)^2
        &= \sum_{i = 1}^n \envert{x_i - z_i}^2 \\
        &= \sum_{i = 1}^n \envert{(x_i - y_i) - (y_i - z_i)}^2 \\
        &= \sum_{i = 1}^n
            \del{
                \envert{x_i - y_i}^2
                - 2 \envert{x_i - y_i} \envert{y_i - z_i}
                + \envert{y_i - z_i}^2
            } \\
        &= \sum_{i = 1}^n \envert{x_i - y_i}^2
           - 2 \sum_{i = 1}^n \envert{x_i - y_i} \envert{y_i - z_i}
           + \sum_{i = 1}^n \envert{y_i - z_i}^2 \\
        &= d_2(x, y)^2
           - 2 \sum_{i = 1}^n \envert{x_i - y_i} \envert{y_i - z_i}
           + d_2(y, z)^2 \\
        &\leq d_2(x, y)^2 + d_2(y, z)^2 \\
        &\leq \del{d_2(x, y) + d_2(y, z)}^2
    .
\end{align*}
%
Taking square roots of both sides of the inequality, we obtain
%
\begin{equation*}
    d_2(x, y) + d_2(y, z) \geq d_2(x, z)
    .
\end{equation*}

\newpage

2. (a) Let $(X, \V)$ be a topological space and $E \subseteq X$. Let
   $E^\prime$ be the set of all limit points of the set $E$. Prove that
   $E^\prime$ is closed.

\begin{proof}

hey

\end{proof}

(b) Consider $(\R^n, d_2)$ with $d_2$ being the Euclidean metric. Is
every point of every nonempty open set $E \subseteq \R^n$ a limit point
of $E$? What if $E$ is closed?

\textit{Solution.}
hey

\newpage

3. Recall that $E^\circ$ denotes the interior of $E$, $\clos{E}$ is the
   closure of $E$, and $E^c$ is the complement of $E$.

(a) Prove that if $G \subseteq E$ and $G$ is open, then $G \subseteq
E^\circ$.

\begin{proof}

hey

\end{proof}

(b) Prove that $\del{E^\circ}^c = \clos{E^c}$.

\begin{proof}

hey

\end{proof}

\end{document}
