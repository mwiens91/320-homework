% Set up the document
\documentclass{article}

% Page size
\usepackage[
    letterpaper,]{geometry}

% Lines between paragraphs
\setlength{\parskip}{\baselineskip}
\setlength{\parindent}{0pt}

% Math
\usepackage{mathtools}
\usepackage{amssymb}
\usepackage{amsthm}
\usepackage{commath}

% Number sets
\newcommand{\Z}{\mathbb{Z}}
\newcommand{\Q}{\mathbb{Q}}
\newcommand{\X}{\mathbb{X}}
\newcommand{\C}{\mathcal{C}}
\newcommand{\N}{\mathbb{N}}
\newcommand{\R}{\mathbb{R}}
\newcommand{\V}{\mathbb{V}}

% Handy shortcuts
\newcommand*\clos[1]{\mkern 1.5mu\overline{\mkern-1.5mu#1\mkern-1.5mu}\mkern 1.5mu}

% Links
\usepackage{hyperref}

% Page numbers at top right
\usepackage{fancyhdr}
\pagestyle{fancy}
\fancyhf{}
\fancyhead[R]{\thepage}
\renewcommand\headrulewidth{0pt}

\begin{document}

\textbf{MATH 320 Worksheet 1} \\
\textbf{Matt Wiens \#301294492} \\
\textbf{2020-01-15}

1. Let $X = \R^n$. Verify that the following are metrics on $X$:

(a)
%
\begin{equation*}
    d_1(x, y) = \sum_{i = 1}^n \envert{x_i - y_i}
    .
\end{equation*}

\textit{Solution.}
We need to verify three properties:

Fix any $x, y, z \in X$.

1. Clearly $d_1(x, y) \geq 0$. Also, we have
%
\begin{align*}
    &d_1(x, y) = 0 \\
    &\iff x_i = y_i, \quad \forall i = 1, \ldots, n \\
    &\iff x = y
    .
\end{align*}
%
2. Here we need to verify the symmetry of $d_1$:
%
\begin{align*}
    d_1(x, y) &= \sum_{i = 1}^n \envert{x_i - y_i} \\
              &= \sum_{i = 1}^n \envert{y_i - x_i} \\
              &= d_1(y, x)
    .
\end{align*}
%
3. For this property we will use the triangle inequality of $|\cdot|$:
%
\begin{align*}
    d_1(x, y) + d_1(y, z)
        &= \sum_{i = 1}^n \envert{x_i - y_i} + \sum_{i = 1}^n \envert{y_i - z_i} \\
        &= \sum_{i = 1}^n \del[2]{\envert{x_i - y_i} + \envert{y_i - z_i}} \\
        &\geq \sum_{i = 1}^n \envert{x_i - z_i} \\
        &= d_1(x, z)
    .
\end{align*}

\vspace{3mm}

(b)
%
\begin{equation*}
    d_2(x, y) = \sqrt{\sum_{i = 1}^n {\envert{x_i - y_i}}^2}
    .
\end{equation*}

\textit{Solution.}
Properties 1 and 2 are verified similarly to what was shown in part (a).
Here we will only verify property 3.

Fix any $x, y, z \in X$. Then
%
\begin{align*}
    d_2(x, z)^2
        &= \sum_{i = 1}^n \envert{x_i - z_i}^2 \\
        &= \sum_{i = 1}^n \envert{(x_i - y_i) - (y_i - z_i)}^2 \\
        &= \sum_{i = 1}^n
            \del{
                \envert{x_i - y_i}^2
                - 2 \envert{x_i - y_i} \envert{y_i - z_i}
                + \envert{y_i - z_i}^2
            } \\
        &= \sum_{i = 1}^n \envert{x_i - y_i}^2
           - 2 \sum_{i = 1}^n \envert{x_i - y_i} \envert{y_i - z_i}
           + \sum_{i = 1}^n \envert{y_i - z_i}^2 \\
        &= d_2(x, y)^2
           - 2 \sum_{i = 1}^n \envert{x_i - y_i} \envert{y_i - z_i}
           + d_2(y, z)^2 \\
        &\leq d_2(x, y)^2 + d_2(y, z)^2 \\
        &\leq \del{d_2(x, y) + d_2(y, z)}^2
    .
\end{align*}
%
Taking square roots of both sides of the inequality, we obtain
%
\begin{equation*}
    d_2(x, y) + d_2(y, z) \geq d_2(x, z)
    .
\end{equation*}

\newpage

2. (a) Let $(X, \V)$ be a topological space and $E \subseteq X$. Let
   $E^\prime$ be the set of all limit points of the set $E$. Prove that
   $E^\prime$ is closed.

\begin{proof}

Let $x$ be a limit point of $E^\prime$. Then for all neighbourhoods
$N_{r_x}(x)$ of $x$, there exists $y \in E^\prime$ ($y \neq x$) such
that $y \in N_{r_x}(x)$. Consider any neighbourhood $N_{r_y}(y) \subset
N_{r_x}(x)$. Since $y \in E^\prime$, there exists $z \in E$ ($z \neq y$)
such that $z \in N_{r_y}(y)$. Thus $z \in N_{r_x}(x)$. Hence we have
shown that for all neighbourhoods $N_{r_x}(x)$ of $x$ there exists $z
\in E$ with $z \neq x$ such that $z \in N_{r_x}(x)$. Therefore $x$ is a
limit point of $E$ and hence $x \in E^\prime$.

Therefore, for every limit point $x$ of $E^\prime$, $x \in E^\prime$, so
$E^\prime$ is closed.

\end{proof}

(b) Consider $(\R^n, d_2)$ with $d_2$ being the Euclidean metric. Is
every point of every nonempty open set $E \subseteq \R^n$ a limit point
of $E$? What if $E$ is closed?

\textit{Solution.}
Let $E \subseteq \R^n$ be open. Then for all $x \in E$, there exists $r
> 0$ such that $N_r(x) \subset E$, and hence for all $l > 0$,
$\del{N_r(x) \setminus \cbr{x}} \cap E \neq \emptyset$. Thus $x$ is a
limit point of $E$.

However, if $E \subseteq \R^n$ is closed, this does not necessarily
hold. Consider the set $E = \cbr{x}$ with $x \in \R^n$. In this case,
$E$ is closed, but has no limit points.

\newpage

3. Recall that $E^\circ$ denotes the interior of $E$, $\clos{E}$ is the
   closure of $E$, and $E^c$ is the complement of $E$.

\textit{Note that my proofs below assume that $E$ has an associated
metric. If $E$ is instead only a topological space, then the below
proofs still work, you just need to refer to neighbourhoods in a more
general way and not characterize them by radii.}

(a) Prove that if $G \subseteq E$ and $G$ is open, then $G \subseteq
E^\circ$.

\begin{proof}

Let $G \subseteq E$ with $G$ open, and let $x \in G$. If $x$ is an
interior point of $E$, then by definition $x \in E^\circ$.

Suppose $x$ is not an interior point of $E$. Then there is no $r > 0$
such that $N_r(x) \subset E$. But this contradicts $G$ being open, for
there must exist $l > 0$ such that $N_l(x) \subset G \subseteq E$. Hence
$x$ must be an interior point of $E$.

Therefore $x \in E^\circ$, and, noting that $x$ was arbitrary, we have
that $G \subseteq E^\circ$.

\end{proof}

(b) Prove that $\del{E^\circ}^c = \clos{E^c}$.

\begin{proof}

Suppose $x \in \del{E^\circ}^c$. Then $x$ is not an interior point of
$E$, so there is no $r > 0$ such that $N_r(x) \subset E$. Therefore, for
all $r > 0$, $N_r(x) \cap E^c \neq \emptyset$, so either $x \in E^c$ or
$x$ is a limit point of $E^c$ (or both). Thus $x \in \clos{E^c}$. This
proves that $\del{E^\circ}^c \subseteq \clos{E^c}$

Now let $x \in \clos{E^c}$. Suppose $x$ is an interior point of $E$.
Then there exists $r > 0$ such that $N_r(x) \subset E$. But since $x \in
\clos{E^c}$, either $x \in E^c$ or for all $l > 0$, $\del{N_l(x)
\setminus \cbr{x}} \cap E^c \neq \emptyset$, both of which contradict
$N_r(x) \subset E$. Thus $x$ is not an interior point of $E$, so $x \in
\del{E^\circ}^c$. Since $x$ was arbitrary, we have $\clos{E^c} \subseteq
\del{E^\circ}^c$.

Therefore, $\del{E^\circ}^c = \clos{E^c}$.

\end{proof}

\end{document}
