% Set up the document
\documentclass{article}

% Page size
\usepackage[
    letterpaper,]{geometry}

% Lines between paragraphs
\setlength{\parskip}{\baselineskip}
\setlength{\parindent}{0pt}

% Math
\usepackage{mathtools}
\usepackage{amssymb}
\usepackage{amsthm}
\usepackage{commath}

% Number sets
\newcommand{\Z}{\mathbb{Z}}
\newcommand{\Q}{\mathbb{Q}}
\newcommand{\X}{\mathbb{X}}
\newcommand{\C}{\mathcal{C}}
\newcommand{\N}{\mathbb{N}}
\newcommand{\R}{\mathbb{R}}
\newcommand{\V}{\mathbb{V}}

% Handy math macros
% For closure bar
\newcommand*\clos[1]{\mkern 1.5mu\overline{\mkern-1.5mu#1\mkern-1.5mu}\mkern 1.5mu}
% For implies arrow with strikethrough
\newcommand{\notimplies}{%
      \mathrel{{\ooalign{\hidewidth$\not\phantom{=}$\hidewidth\cr$\implies$}}}}

% Links
\usepackage{hyperref}

% Page numbers at top right
\usepackage{fancyhdr}
\pagestyle{fancy}
\fancyhf{}
\fancyhead[R]{\thepage}
\renewcommand\headrulewidth{0pt}

\begin{document}

\textbf{MATH 320 Homework 1} \\
\textbf{Matt Wiens \#301294492} \\
\textbf{2020-01-17}

1. For $x, y \in \R$, define
%
\begin{align}
    d_1(x, y) &= (x - y)^2,
    \label{eq:q1-1} \\
    d_2(x, y) &= \sqrt{\, \envert{x - y}},
    \label{eq:q1-2} \\
    d_3(x, y) &= \envert{x^2 - y^2},
    \label{eq:q1-3} \\
    d_4(x, y) &= \envert{x - 2 y}.
    \label{eq:q1-4}
\end{align}
%
For each of these, determine whether it is a metric or not.

\textit{Solution.}
For each of these functions we will determine if they are metrics  by
checking they satisfy the three properties discussed in class. Let $x,
y, z \in \R$.

For the first property, clearly $d_1, d_2, d_3, d_4 \geq 0$.
Additionally, for $d_1$ we have that
%
\begin{align*}
    &d_1(x, y) = 0 \\
    &\iff (x - y)^2 = 0 \\
    &\iff x - y = 0 \\
    &\iff x = y
    ;
\end{align*}
%
and similarly for $d_2$ we have that
%
\begin{align*}
    &d_2(x, y) = 0 \\
    &\iff \sqrt{\, \envert{x - y}} = 0 \\
    &\iff \envert{x - y} = 0 \\
    &\iff x = y
    .
\end{align*}
%
Thus $d_1$ and $d_2$ satisfy the first property. However for $d_3$ we
have
%
\begin{align*}
    &d_3(x, y) = 0 \\
    &\iff \envert{x^2 - y^2} = 0 \\
    &\iff x^2 = y^2 \\
    &\notimplies x = y
    .
\end{align*}
%
Therefore $d_3$ is not a metric.

For $d_4$ we have
%
\begin{align*}
    &d_4(x, y) = 0 \\
    &\iff \envert{x - 2 y} = 0 \\
    &\iff x = 2 y \\
    &\notimplies x = y
    ,
\end{align*}
%
so $d_4$ is also not a metric.

Now we will check whether $d_1$ and $d_2$ satisfy the second property.
Since
%
\begin{align*}
    d_1(x, y) &= (x - y)^2 \\
              &= (y - x)^2 \\
              &= d_1(y, x),
\end{align*}
%
and
%
\begin{align*}
    d_2(x, y) &= \sqrt{\, \envert{x - y}} \\
              &= \sqrt{\, \envert{y - x}} \\
              &= d_2(y, x),
\end{align*}
%
both $d_1$ and $d_2$ satisfy the second property.

Now we will check the third property. For $d_1$, consider the case when
$x = 1$, $y = 0.1$, and $z = 0$. Then, in this case
%
\begin{align*}
    d_1(x, y) + d_1(y, z)
        &= d_1(1, 0.1) + d_1(0.1, 0) \\
        &= (1 - 0.1)^2 + (0.1 - 0)^2 \\
        &= 0.81 + 0.01 \\
        &= 0.82 \\
        &< 1 \\
        &= (1 - 0)^2 \\
        &= d_1(1, 0) \\
        &= d_1(x, z)
        .
\end{align*}
%
This counterexample shows that $d_1$ is not a metric.

For $d_2$, we will the triangle inequality of $|\cdot|$ to show that
$d_2$ satisfies the third property:
%
\begin{align*}
    d_2(x, z)^2
        &= \envert{x - z} \\
        &\leq \envert{x - y} + \envert{y - z} \\
        &= d_2(x, y)^2 + d_2(y, z)^2 \\
        &\leq \del{d_2(x, y) + d_2(y, z)}^2
        .
\end{align*}
%
Taking square roots of both sides, we obtain
%
\begin{equation*}
    d_2(x, y) + d_2(y, z) \geq d_2(x, z)
    .
\end{equation*}
%
Therefore $d_2$ also satisfies the third property and hence is a metric.
As was shown above, neither $d_1$, $d_3$, nor $d_4$ are metrics.

\newpage

2. (a) Suppose $(X, d)$ is a metric space. Prove that $\tilde{d}$
   defined below defines another metric on $X$:
%
\begin{equation*}
    \tilde{d} (x, y) = \frac{d(x,y)}{1 + d(x,y)}
    .
\end{equation*}

\begin{proof}

We will prove that $\tilde{d}$ satisfies all three properties required
for it to be a metric. Let $x, y, z \in X$.

For the first property, clearly $\tilde{d}(x, y) \geq 0$. Additionally,
%
\begin{align*}
    &\tilde{d}(x, y) = 0 \\
    &\iff \frac{d(x,y)}{1 + d(x,y)} = 0 \\
    &\iff d(x,y) = 0 \\
    &\iff x = y
    .
\end{align*}
%
Thus, $\tilde{d}$ satisfies the first property.

For the second property, we have
%
\begin{align*}
    \tilde{d}(x, y)
        &= \frac{d(x,y)}{1 + d(x,y)} \\
        &= \frac{d(y,x)}{1 + d(y,x)} \\
        &= \tilde{d}(y, x)
    ,
\end{align*}
%
so $\tilde{d}$ also satisfies the second property.

For the third property, we have that
%
\begin{align*}
    \tilde{d}(x, z)
        &= \frac{d(x,z)}{1 + d(x,z)} \\
        &= \frac{1}{1 + \frac{1}{d(x,z)}} \\
        &\leq \frac{1}{1 + \frac{1}{d(x,y) + d(y,z)}} \\
        &= \frac{d(x,y) + d(y, z)}{1 + d(x,y) + d(y,z)} \\
        &= \frac{d(x,y)}{1 + d(x,y) + d(y,z)} + \frac{d(y, z)}{1 + d(x,y) + d(y,z)} \\
        &\leq \frac{d(x,y)}{1 + d(x,y)} + \frac{d(y, z)}{1 + d(y,z)} \\
        &= \tilde{d}(x, y) + \tilde{d}(y, z)
        .
\end{align*}
%
Thus $\tilde{d}$ satisfies the third property.

Since we have shown that $\tilde{d}$ satisfies all three properties
required of a metric, we conclude that $\tilde{d}$ is a metric on $X$.

\end{proof}

(b) Suppose $Y$ is a set and $(X, d_X)$ is a metric space. Let $f: Y \to
X$ be an injection. Prove that $f$ induces a metric on $Y$ by
%
\begin{equation*}
    d_Y (a, b) = d_X(f(a), f(b))
    .
\end{equation*}

\begin{proof}

We will prove that $d_Y$ satisfies all three properties of a metric. Let
$a, b, c \in Y$.

For the first property, clearly $d_Y(a, b) \geq 0$. Additionally
%
\begin{align*}
    &d_Y(a, b) = 0 \\
    &\iff d_X(f(a), f(b)) = 0 \\
    &\iff f(a) = f(b) \\
    &\iff a = b
    ,
\end{align*}
%
where the last equivalence holds because $f$ is injective.

For the second property, we have
%
\begin{align*}
    d_Y(a, b) &= d_X(f(a), f(b)) \\
              &= d_X(f(b), f(a)) \\
              &= d_Y(b, a)
              .
\end{align*}
%
Finally, for the third property, we have that
%
\begin{align*}
    d_Y(a, c) &= d_X(f(a), f(c)) \\
              &\leq d_X(f(a), f(b)) + d_X(f(b), f(c)) \\
              &= d_Y(a, b) + d_Y(b, c)
              .
\end{align*}
%
Since $d_Y$ satisfies all of the three properties, it induces a metric
on $Y$.

\end{proof}

\newpage

3. Let $(X, d)$ be a metric space and $E \subseteq X$. Recall the
   notations: $E^\circ$ is the interior of $E$ and $\clos{E}$ is the
   closure of $E$. Prove or disprove (by showing a counterexample) the
   following:

   (a) $E$ and $\clos{E}$ always have the same interior.

\textit{Solution.}
This is false. Let $X = \R$ and consider $E = (-1, 0) \cup (0, 1)$. Then
$E^\circ = E$. However, $\clos{E} = [-1, 1]$, so
$\del[1]{\clos{E}}^\circ = (-1, 1) \neq E^\circ$.

(b) $E$ and $E^\circ$ always have the same closure.

\textit{Solution.}
This is also false. Let $X = \R$ and let $E = \cbr{x}$ where $x \in \R$.
Then $\clos{E} = E$. However, $E^\circ = \emptyset$, so $\clos{E^\circ}
= \emptyset \neq \clos{E}$.

\newpage

4. Let $C[-1, 1]$ be the set of functions that are continuous on $[-1,
   1]$. Define the metric on $C[-1, 1]$ as
%
\begin{align*}
    d(f, g) =  \sup_{[-1, 1]} |f - g|,
    \qquad
    f, g \in C[-1, 1].
\end{align*}
%
Let $E \subseteq C[-1, 1]$ be the set given by
%
\begin{align*}
   E = \{f \in C[-1, 1] : \text{$f$ is differentiable on (-1, 1)}\}.
\end{align*}

Is $E$ a closed subset of the space $(C[-1, 1], d)$? Justify your answer
by either proving it or showing a counterexample.

\textit{Solution.}
$E$ is not a closed subset of the space. To show why this is so, we will
use the Stone-Weierstrass thoerem (SE), which guarantees that for any $f
\in C([-1, 1])$ and any $r > 0$, there exists a polynomial $p$ such that
$\sup_{[-1, 1]} \envert{f - p} < r$. Recall that polynomials are
differentiable, so in the above we would have that $p \in E$.

Let $f \in C[-1, 1] \setminus E$. Then for any $r > 0$, using the SE
theorem we have that $\del{N_r(f) \setminus \cbr{f}} \cap E \neq
\emptyset$. Therefore $f$ is a limit point of $E$, but $f \not\in E$, so
$E$ is not closed.

\end{document}
