% Set up the document
\documentclass{article}

% Page size
\usepackage[
    letterpaper,]{geometry}

% Lines between paragraphs
\setlength{\parskip}{\baselineskip}
\setlength{\parindent}{0pt}

% Math
\usepackage{mathtools}
\usepackage{amssymb}
\usepackage{amsthm}
\usepackage{commath}

% Number sets
\newcommand{\Z}{\mathbb{Z}}
\newcommand{\Q}{\mathbb{Q}}
\newcommand{\X}{\mathbb{X}}
\newcommand{\C}{\mathcal{C}}
\newcommand{\N}{\mathbb{N}}
\newcommand{\R}{\mathbb{R}}
\newcommand{\V}{\mathbb{V}}

% Handy shortcuts
\newcommand*\clos[1]{\mkern 1.5mu\overline{\mkern-1.5mu#1\mkern-1.5mu}\mkern 1.5mu}

% Links
\usepackage{hyperref}

% Page numbers at top right
\usepackage{fancyhdr}
\pagestyle{fancy}
\fancyhf{}
\fancyhead[R]{\thepage}
\renewcommand\headrulewidth{0pt}

\begin{document}

\textbf{MATH 320 Homework 1} \\
\textbf{Matt Wiens \#301294492} \\
\textbf{2020-01-17}

1. For $x, y \in \R$, define
%
\begin{align}
    d_1(x, y) &= (x - y)^2,
    \label{eq:q1-1} \\
    d_2(x, y) &= \sqrt{\, \envert{x - y}},
    \label{eq:q1-2} \\
    d_3(x, y) &= \envert{x^2 - y^2},
    \label{eq:q1-3} \\
    d_4(x, y) &= \envert{x - 2 y}.
    \label{eq:q1-4}
\end{align}
%
For each of these, determine whether it is a metric or not.

\textit{Solution.}
hey

\newpage

2. (a) Suppose $(X, d)$ is a metric space. Prove that $\tilde{d}$
   defined below defines another metric on $X$:
%
\begin{equation*}
    \tilde{d} (x, y) = \frac{d(x,y)}{1 + d(x,y)}
    .
\end{equation*}

\begin{proof}

hey

\end{proof}

(b) Suppose $Y$ is a set and $(X, d_X)$ is a metric space. Let $f: Y \to
X$ be an injection. Prove that $f$ induces a metric on $Y$ by
%
\begin{equation*}
    d_Y (a, b) = d_X(f(a), f(b))
    .
\end{equation*}

\begin{proof}

hey

\end{proof}

\newpage

3. Let $(X, d)$ be a metric space and $E \subseteq X$. Recall the
   notations: $E^\circ$ is the interior of $E$ and $\clos{E}$ is the
   closure of $E$. Prove or disprove (by showing a counterexample) the
   following:

   (a) $E$ and $\clos{E}$ always have the same interior.

\begin{proof}

hey

\end{proof}

(b) $E$ and $E^\circ$ always have the same closure.

\begin{proof}

hey

\end{proof}

\newpage

4. Let $C[-1, 1]$ be the set of functions that are continuous on $[-1,
   1]$. Define the metric on $C[-1, 1]$ as
%
\begin{align*}
    d(f, g) =  \sup_{[-1, 1]} |f - g|,
    \qquad
    f, g \in C[-1, 1].
\end{align*}
%
Let $E \subseteq C[-1, 1]$ be the set given by
%
\begin{align*}
   E = \{f \in C[-1, 1] : \text{$f$ is differentiable on (-1, 1)}\}.
\end{align*}

Is $E$ a closed subset of the space $(C[-1, 1], d)$? Justify your answer
by either proving it or showing a counterexample.

\textit{Solution.}
hey

\end{document}
