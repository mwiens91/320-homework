% Set up the document
\documentclass{article}

% Page size
\usepackage[
    letterpaper,]{geometry}

% Lines between paragraphs
\setlength{\parskip}{\baselineskip}
\setlength{\parindent}{0pt}

% Math
\usepackage{mathtools}
\usepackage{amssymb}
\usepackage{amsthm}
\usepackage{commath}

% Operators
\DeclareMathOperator{\diam}{diam}

% Number sets
\newcommand{\C}{\mathcal{C}}
\newcommand{\N}{\mathbb{N}}
\newcommand{\Q}{\mathbb{Q}}
\newcommand{\R}{\mathbb{R}}
\newcommand{\Z}{\mathbb{Z}}

% Links
\usepackage{hyperref}

% Page numbers at top right
\usepackage{fancyhdr}
\pagestyle{fancy}
\fancyhf{}
\fancyhead[R]{\thepage}
\renewcommand\headrulewidth{0pt}

\begin{document}

\textbf{MATH 320 Worksheet 4} \\
\textbf{Matt Wiens \#301294492} \\
\textbf{2020-02-27}

1. Let $\sum_{n = 1}^\infty a_n$ be a convergent series and
   $\cbr{b_n}_{n = 1}^\infty$ be a bounded sequence.

(a) Show that if $\sum_{n = 1}^\infty a_n$ converges absolutely, then
$\sum_{n = 1}^\infty a_n b_n$ also converges absolutely.

\begin{proof}

\end{proof}

(b) Suppose $\sum_{n = 1}^\infty a_n$ converges conditionally. Is it
true that $\sum_{n = 1}^\infty a_n b_n$ must converge? Prove or disprove
by showing a counterexample.

\textit{Solution.}

\newpage

2. Suppose that the coefficients of the power series $\sum_{n =
   1}^\infty a_n x^n$ are integers, infinitely many of which are
   distinct from zero. Prove that the radius of convergence is at most
   $1$.

\begin{proof}

\end{proof}

\newpage

3. Suppose $a_n \geq 0$ and $\sum_{n = 1}^\infty a_n$ converges.

(a) Show that
%
\begin{equation*}
    \sum_{n = 1}^\infty \sqrt{\frac{a_n}{n^{1 + \delta}}}
\end{equation*}
%
must converge if $\delta > 0$.

\begin{proof}

\end{proof}

(b) What happens if $\delta = 0$?

\textit{Solution.}
hey

\end{document}
