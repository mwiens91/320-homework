% Set up the document
\documentclass{article}

% Page size
\usepackage[
    letterpaper,]{geometry}

% Lines between paragraphs
\setlength{\parskip}{\baselineskip}
\setlength{\parindent}{0pt}

% Math
\usepackage{mathtools}
\usepackage{amssymb}
\usepackage{amsthm}
\usepackage{commath}

% Operators
\DeclareMathOperator{\diam}{diam}

% Number sets
\newcommand{\C}{\mathcal{C}}
\newcommand{\N}{\mathbb{N}}
\newcommand{\Q}{\mathbb{Q}}
\newcommand{\R}{\mathbb{R}}
\newcommand{\Z}{\mathbb{Z}}

% Links
\usepackage{hyperref}

% Page numbers at top right
\usepackage{fancyhdr}
\pagestyle{fancy}
\fancyhf{}
\fancyhead[R]{\thepage}
\renewcommand\headrulewidth{0pt}

\begin{document}

\textbf{MATH 320 Worksheet 4} \\
\textbf{Matt Wiens \#301294492} \\
\textbf{2020-02-27}

1. Let $\sum_{n = 1}^\infty a_n$ be a convergent series and
   $\cbr{b_n}_{n = 1}^\infty$ be a bounded sequence.

(a) Show that if $\sum_{n = 1}^\infty a_n$ converges absolutely, then
$\sum_{n = 1}^\infty a_n b_n$ also converges absolutely.

\begin{proof}

Since the sequence $\cbr{b_n}$ is bounded there exists some $M > 0$ such
that for all $n$, $|b_n| < M$. Then we have that
%
\begin{equation*}
    \sum_{n = 1}^\infty | a_n b_n |
        \leq M \sum_{n = 1}^\infty | a_n |
        < \infty
        ,
\end{equation*}
%
since the product of any two finite numbers if finite. Hence $\sum_{n =
1}^\infty a_n b_n$ also converges absolutely.

\end{proof}

(b) Suppose $\sum_{n = 1}^\infty a_n$ converges conditionally. Is it
true that $\sum_{n = 1}^\infty a_n b_n$ must converge? Prove or disprove
by showing a counterexample.

\textit{Solution.}
In this case it is not true that $\sum_{n = 1}^\infty a_n b_n$ must converge.
Take
%
\begin{equation*}
    a_n = (-1)^{n + 1} \frac{1}{n}
    .
\end{equation*}
%
From lectures we know that $\sum_{n = 1}^\infty a_n$ converges
conditionally (it is the alternating harmonic series). However, if we
take
%
\begin{equation*}
    b_n = (-1)^{n + 1}
    ,
\end{equation*}
%
then $\sum_{n = 1}^\infty a_n b_n$ is the harmonic series, which
diverges.

\newpage

2. Suppose that the coefficients of the power series $\sum_{n =
   1}^\infty a_n x^n$ are integers, infinitely many of which are
   distinct from zero. Prove that the radius of convergence is at most
   $1$.

\begin{proof}

Since infinitely many of the $a_n$s are non-zero, there exists $N_1 > 0$ such
that for all $n > N_1$, $a_n \neq 0$. Using our formula for for the radius
of convergence $R_0$, we have
%
\begin{equation*}
    R_0 = \frac{1}{\limsup\limits_{n \to \infty} \envert{\frac{a_{n + 1}}{a_n}}}
    .
\end{equation*}
%
Note that this formula is valid for $n > N_1$, which is implicit in the
limit $n \to \infty$. Suppose $R_0 > 1$. Then we must have that
%
\begin{equation*}
    \limsup_{n \to \infty} \envert{\frac{a_{n + 1}}{a_n}} < 1
    ,
\end{equation*}
%
which implies that for some $N_2 > 0$, for all $n > N_2$
%
\begin{equation*}
    |a_{n + 1}| < |a_n|
    .
\end{equation*}
%
However, since $a_{N_2}$ is finite, and $a_n \in \Z$, the above
inequality can only hold for finitely many $n > N_2$. Hence we have
arrived at a contradiction, and conclude that $R_0 \leq 1$.

\end{proof}

\newpage

3. Suppose $a_n \geq 0$ and $\sum_{n = 1}^\infty a_n$ converges.

(a) Show that
%
\begin{equation*}
    \sum_{n = 1}^\infty \sqrt{\frac{a_n}{n^{1 + \delta}}}
\end{equation*}
%
must converge if $\delta > 0$.

\begin{proof}

Here we will use the Cauchy-Schwarz inequality as follows:
%
\begin{align*}
    \del{\sum_{n = 1}^\infty \sqrt{\frac{a_n}{n^{1 + \delta}}}}^2
        &\leq
            \sum_{n = 1}^\infty \del{\sqrt{{a_n}}}^2
            \cdot
            \sum_{n = 1}^\infty \del{\sqrt{\frac{1}{n^{1 + \delta}}}}^2
            \\
        &=
            \sum_{n = 1}^\infty {a_n}
            \cdot
            \sum_{n = 1}^\infty \frac{1}{n^{1 + \delta}}
            \\
        &< \infty
        ,
\end{align*}
%
since $\sum_{n = 1}^\infty {a_n}$ converges and by assumption of $\delta
> 0$, $\sum_{n = 1}^\infty \frac{1}{n^{1 + \delta}}$ also converges. Hence
%
\begin{equation*}
    \sum_{n = 1}^\infty \sqrt{\frac{a_n}{n^{1 + \delta}}} < \infty
\end{equation*}
%
and $\sum_{n = 1}^\infty \sqrt{\frac{a_n}{n^{1 + \delta}}}$ is convergent.

\end{proof}

(b) What happens if $\delta = 0$?

\textit{Solution.}
Now we have no guarantee that $\sum_{n = 1}^\infty \sqrt{\frac{a_n}{n
}}$ is convergent. As a counterexample, suppose
%
\begin{equation*}
    a_n = \frac{1}{n (\log n)^2}
    .
\end{equation*}
%
From lectures we know that $\sum_{n = 1}^\infty a_n$ is convergent. However,
%
\begin{equation*}
    \sqrt{\frac{a_n}{n}} = \frac{1}{n \log n}
    ,
\end{equation*}
%
and, also from lectures, we know that the corresponding series $\sum_{n
= 1}^\infty \sqrt{\frac{a_n}{n }}$ diverges.


\end{document}
