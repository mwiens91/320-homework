% Set up the document
\documentclass{article}

% Page size
\usepackage[
    letterpaper,]{geometry}

% Lines between paragraphs
\setlength{\parskip}{\baselineskip}
\setlength{\parindent}{0pt}

% Math
\usepackage{mathtools}
\usepackage{amssymb}
\usepackage{amsthm}
\usepackage{commath}

% Operators
\DeclareMathOperator{\diam}{diam}

% Number sets
\newcommand{\C}{\mathcal{C}}
\newcommand{\N}{\mathbb{N}}
\newcommand{\Q}{\mathbb{Q}}
\newcommand{\R}{\mathbb{R}}
\newcommand{\Z}{\mathbb{Z}}

% Links
\usepackage{hyperref}

% Page numbers at top right
\usepackage{fancyhdr}
\pagestyle{fancy}
\fancyhf{}
\fancyhead[R]{\thepage}
\renewcommand\headrulewidth{0pt}

\begin{document}

\textbf{MATH 320 Worksheet 3} \\
\textbf{Matt Wiens \#301294492} \\
\textbf{2020-02-13}

1. (a) Consider the metric space $(C[0, 1], d_\infty)$ with
%
\begin{equation*}
    d_\infty(f, g) = \sup_{x \in [0, 1]} |f(x) - g(x)|
    .
\end{equation*}
%
Let $\cbr{f_n}_{n = 1}^\infty$ be the sequence of functions given by
%
\begin{equation*}
    f_n(x) = \frac{1}{n x + 1}, \quad x \in [0, 1].
\end{equation*}
%
Show by the definition of Cauchy sequences that $\cbr{f_n}_{n =
1}^\infty$ is not a Cauchy sequence in the space $(C[0, 1], d_\infty)$.

\begin{proof}

Suppose for a contradiction that $\cbr{f_n}_{n = 1}^\infty$ was Cauchy.
Then for $\epsilon = 0.1$ there would be $N \in \N$ such that for all
$m, n \geq N$, we would have $d_\infty(f_n, f_m) < \epsilon$. Fix $n = N$.

Consider that
%
\begin{equation*}
    \envert{f_m(x) - f_N(x)}
        = \frac{1}{N x + 1} - \frac{1}{m x + 1}
        = \frac{(m - N) x}{(N x + 1) (m x + 1)}
        .
\end{equation*}
%
Suppse $x = x_m = \frac{1}{m}$. Then
%
\begin{equation*}
    \envert{f_m(x_m) - f_N(x_m)}
        = \frac{1 - \frac{N}{m}}{2 (\frac{N}{m} + 1)}
        ;
\end{equation*}
%
if we take the limit as $m \to \infty$ then we clearly have that
%
\begin{equation*}
    \envert{f_m(x_m) - f_N(x_m)} \to \frac{1}{2}
    .
\end{equation*}
%
Hence by setting $m$ sufficiently high, we can make this difference
arbitrary close to $1 / 2$, and thus for this same $m$ we would have
that
%
\begin{equation*}
    d_\infty(f_N, f_m) = \sup_{x \in [0, 1]} |f_N(x) - f_m(x)| > 0.1 = \epsilon
    .
\end{equation*}
%
This contradicts our assumption that $\cbr{f_n}_{n = 1}^\infty$ was Cauchy.

\end{proof}

\newpage

(b) Instead consider $(C[1, 2], d_\infty)$. Is $\cbr{f_n}_{n =
1}^\infty$ a Cauchy sequence?

\textit{Solution.}
In this case the sequence $\cbr{f_n}_{n = 1}^\infty$ would be Cauchy.
For any two elements of this sequence, say $f_n$ and $f_m$ we have that
, without loss of generality taking $m \geq n$,
%
\begin{align*}
    d_\infty(f_n, f_m)
        &= \sup_{x \in [1, 2]} |f(x) - g(x)| \\
        &= \sup_{x \in [1, 2]} \frac{(m - N) x}{(n x + 1) (m x + 1)} \\
        &= \sup_{x \in [1, 2]} \frac{(\frac{1}{n} - \frac{1}{m}) x}{(x + \frac{1}{n}) (x + \frac{1}{m})}
    .
\end{align*}
%
Since $x \in [1, 2]$ we can make $(\frac{1}{n} - \frac{1}{m}) x$
arbitrarily small by setting $n$ and $m$ sufficiently high. Since the
denominator of the above expression is bounded (it's guaranteed less
than $25$), it follows we can make the entire expression arbitrarily
small. Hence in this space, the sequence is Cauchy.

\newpage

2. Let $(X, d)$ be a nonempty metric space with $d$ being the discrete
metric. Is $(X, d)$ complete?

\textit{Solution.}
This space is complete. Let $S = \cbr{x_n}$ be any Cauchy sequence in
$(X, d)$. Fix any $\epsilon > 0$ with $\epsilon < 1$. Then there exists
an $N \in \N$ such that for all $m, n \geq N$, $d(x_n, x_m) < \epsilon <
1$. But this implies that for all such $m, n$, $x_n = x_m$. Call this
value $x$.

Then for any $\epsilon > 0$, if $n \geq N$, $d(x_n, x) = 0 < \epsilon$
so $S$ is a convergent sequence. Since $S$ was an arbitrary Cauchy
sequence, it follows that $(X, d)$ is complete.

\newpage

3. Let $E_n$ be a sequence of nonempty, closed, and bounded sets in a
complete metric space $(X, d)$. Show that if $E_n \supseteq E_{n+1}$ and
%
\begin{equation*}
    \lim_{n \to \infty} \diam E_n = 0,
\end{equation*}
%
Then $\cap_{n = 1}^{\infty} E_n$ consists of exactly one point.

\begin{proof}

Denote $E = \cap_{n = 1}^{\infty} E_n$. First we will show that $E$ is
nonempty. Let $S = \cbr{x_n}$, where $x_n \in E_n$. Fix any $\epsilon >
0$. Then there exists some $N$ for which $\diam E_N < \epsilon$. Then,
for all $m, n \geq N$ we have that $d(x_m, x_n) < \epsilon$. Hence $S$
is Cauchy.

Since $\cbr{x_n}$ is Cauchy, it converges in this space to some $x$. By
the definition of our sequence, we must have that $x$ is either in all
$E_n$ or is a limit point of these sets. But since all $E_n$ are closed,
we must have that $x \in E_n$ for all $n$, and hence $x \in E$.

That $E$ has only one point follows trivially from $\lim_{n \to \infty} \diam E_n = 0$.

\end{proof}

\end{document}
