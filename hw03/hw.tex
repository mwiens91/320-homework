% Set up the document
\documentclass{article}

% Page size
\usepackage[
    letterpaper,]{geometry}

% Lines between paragraphs
\setlength{\parskip}{\baselineskip}
\setlength{\parindent}{0pt}

% Math
\usepackage{mathtools}
\usepackage{amssymb}
\usepackage{amsthm}
\usepackage{commath}

% Number sets
\newcommand{\C}{\mathcal{C}}
\newcommand{\N}{\mathbb{N}}
\newcommand{\Q}{\mathbb{Q}}
\newcommand{\R}{\mathbb{R}}
\newcommand{\Z}{\mathbb{Z}}

% Links
\usepackage{hyperref}

% Page numbers at top right
\usepackage{fancyhdr}
\pagestyle{fancy}
\fancyhf{}
\fancyhead[R]{\thepage}
\renewcommand\headrulewidth{0pt}

\begin{document}

\textbf{MATH 320 Homework 3} \\
\textbf{Matt Wiens \#301294492} \\
\textbf{2020-01-31}

1. Consider the usual Euclidean metric space $(\R, d_2)$. For each $n
   \in \N$, we define the function
%
\begin{equation*}
    f_n(x) \coloneqq
        \begin{cases}
            1 ,& n \leq x < n + 1 \\
            0 ,& \text{otherwise}
        \end{cases}
    .
\end{equation*}
%
Define the sequence $\cbr{c_n}_{n \in \N}$ by
%
\begin{equation*}
    c_n \coloneqq \int_{\R} f_n(x) \dif x
    .
\end{equation*}
%
Does the sequence $\cbr{c_n}$ converge? What is the limit?

\textit{Solution.}
Note that for each $n \in \N$,
%
\begin{equation*}
    c_n = \int_{\R} f_n(x) \dif x = \int_{n}^{n + 1} \dif x = 1
    .
\end{equation*}
%
Hence $\cbr{c_n}$ is a constant sequence whose members are all $1$;
therefore, it clearly converges where
%
\begin{equation*}
    \lim_{n \to \infty} c_n = 1
    .
\end{equation*}

\newpage

2. Consider the same metric space and functions $f_n$ as in the previous
   question. Fix $x^* \in \R$, and denote
%
\begin{equation*}
    f(x^*) \coloneqq \lim_{n \to \infty} f_n(x^*)
    .
\end{equation*}
%
What is $f(x^*)$? What is $\int_{\R} f(x^*) \dif x$?

\textit{Solution.} Fix $x^* \in \R$. Then we that
%
\begin{equation*}
    f(x^*) = \lim_{n \to \infty} f_n(x^*) = 0
    ,
\end{equation*}
%
since there exists $m \in \N$ such that $m > x^*$, and thus for all $n
\geq m$, $f_n(x^*) = 0$. Therefore
%
\begin{equation*}
    \int_{\R} f(x^*) \dif x
    = \int_{\R} 0 \dif x
    = 0
    .
\end{equation*}

\newpage

3. Let $X$ denote the set of real sequences $\cbr{a_n}$ for which
   $\sum_{n = 1}^\infty |a_n| < +\infty$. Let $d_1$ and $d_{\infty}$ be
   the metrics on $X$ given as
%
\begin{align*}
    d_1(x,y) &\coloneqq \sum_{n = 1}^\infty |x_n-y_n|, \\
    d_{\infty}(x,y) &\coloneqq \sup_{n \in \N} |x_n - y_n|,
\end{align*}
%
for all $x = \cbr{x_n}, y = \cbr{y_n} \in X$.

(a) Confirm these are both metrics.

\begin{proof}

We will show that both $d_1$ and $d_\infty$ are metrics by verifying
that they each satisfy all three properties sufficient for them to be
metrics. Fix $x = \cbr{x_n}, y = \cbr{y_n}, z = \cbr{z_n} \in X$.

We'll first consider $d_1$.

\textbf{Property 1}

Clearly $d_1(x, y) \geq 0$. Additionally, we have that
%
\begin{align*}
    &d_1(x, y) = 0 \\
    &\iff \sum_{n = 1}^\infty \envert{x_n - y_n} = 0 \\
    &\iff \envert{x_n - y_n} = 0, \quad \forall n = 1, 2, \ldots \\
    &\iff x_n = y_n, \quad \forall n = 1, 2, \ldots \\
    &\iff x = y
    .
\end{align*}
%
\textbf{Property 2}

To show that $d_1$ is symmetric, consider that
%
\begin{align*}
    d_1(x, y)
        &= \sum_{n = 1}^\infty |x_n - y_n| \\
        &= \sum_{n = 1}^\infty |y_n - x_n| \\
        &= d_1(y, x)
        .
\end{align*}
%
\textbf{Property 3}

Now we need to show that $d_1$ satisfies the triangle inequality.
Here we will use the triangle inequality of $|\cdot|$:
%
\begin{align*}
    d_1(x, y) + d_1(y, z)
        &= \sum_{n = 1}^\infty |x_n - y_n| + \sum_{n = 1}^\infty |y_n - z_n| \\
        &= \sum_{n = 1}^\infty \del{|x_n - y_n| + |y_n - z_n|} \\
        &\geq \sum_{n = 1}^\infty |x_n - z_n| \\
        &= d_1(x, z)
        .
\end{align*}
%
Since $d_1$ satisfies all three properties, $d_1$ is a metric on $X$.

Now we'll consider $d_2$.

\textbf{Property 1}

Clearly $d_2(x, y) \geq 0$. Additionally, we have that
%
\begin{align*}
    &d_2(x, y) = 0 \\
    &\iff \sup_{n \in \N} |x_n - y_n| = 0 \\
    &\iff x_n = y_n, \quad \forall n = 1, 2, \ldots \\
    &\iff x = y
    .
\end{align*}
%
\textbf{Property 2}

To show that $d_2$ is symmetric, consider that
%
\begin{align*}
    d_2(x, y)
        &\iff \sup_{n \in \N} |x_n - y_n| = 0 \\
        &\iff \sup_{n \in \N} |y_n - x_n| = 0 \\
        &= d_2(y, x)
        .
\end{align*}
%
\textbf{Property 3}

Now we need to show that $d_2$ satisfies the triangle inequality.
%Here we will use the triangle inequality of $|\cdot|$:
%%
%\begin{align*}
%    d_1(x, y) + d_1(y, z)
%        &= \sum_{n = 1}^\infty |x_n - y_n| + \sum_{n = 1}^\infty |y_n - z_n| \\
%        &= \sum_{n = 1}^\infty \del{|x_n - y_n| + |y_n - z_n|} \\
%        &\geq \sum_{n = 1}^\infty |x_n - z_n| \\
%        &= d_1(x, z)
%        .
%\end{align*}
%%
%Since $d_1$ satisfies all three properties, $d_1$ is a metric on $X$.

\end{proof}

(b) Show that there are sequences $\cbr{x_k}$ in $X$ which converge in
$(X, d_{\infty})$ but fail to converge in $(X, d_1)$.

\begin{proof}
\end{proof}

(c) Show that if $\cbr{x_n}$ converges in $(X, d_1)$ then it
automatically converges in $(X, d_{\infty})$.

\begin{proof}
\end{proof}

(d) If $(X, d)$ is a complete metric space, prove that $(X, \tilde{d})$
is also complete, where $\tilde{d} \coloneqq \frac{d}{1 + d}$. Prove
that if $(X, \tilde{d})$ is complete, then so is $(X, d)$.

\begin{proof}
\end{proof}

\end{document}
