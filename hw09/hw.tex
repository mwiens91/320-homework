% Set up the document
\documentclass{article}

% Page size
\usepackage[
    letterpaper,]{geometry}

% Lines between paragraphs
\setlength{\parskip}{\baselineskip}
\setlength{\parindent}{0pt}

% Math
\usepackage{mathtools}
\usepackage{amssymb}
\usepackage{amsthm}
\usepackage{commath}

% Number sets
\newcommand{\C}{\mathcal{C}}
\newcommand{\N}{\mathbb{N}}
\newcommand{\Q}{\mathbb{Q}}
\newcommand{\R}{\mathbb{R}}
\newcommand{\Z}{\mathbb{Z}}

% Links
\usepackage{hyperref}

% Page numbers at top right
\usepackage{fancyhdr}
\pagestyle{fancy}
\fancyhf{}
\fancyhead[R]{\thepage}
\renewcommand\headrulewidth{0pt}

\begin{document}

\textbf{MATH 320 Homework 9} \\
\textbf{Matt Wiens \#301294492} \\
\textbf{2020-04-14}

Consider the equation
%
\begin{equation}
    \dod{u}{t} = u^2 - 2 u^3 + 2, \qquad \eval[1]{u}_{t = 0} = 1
    .
    \label{eq:main}
\end{equation}
%
Show that there exists $t_0 > 0$ such that~\eqref{eq:main}
has a unique solution $u(t)$ for $t \in [0, t_0]$.

\vspace{5mm}

\textbf{Main setting.} Convert~\eqref{eq:main} into its integral form by
integrating both sides in $t$:
%
\begin{align*}
    u(t) &= u(0) + \int_0^t \del{u^2(s) - 2 u^3(s) + 2} \dif s \\
         & = 1 + \int_0^t \del{u^2(s) - 2 u^3(s) + 2} \dif s
         .
\end{align*}
%
Define the operator $T: C([0, t_0]) \to C([0, t_0])$ (with $t_0$ to be
determined) as
%
\begin{equation*}
   Tw = 1 + \int_0^t \del{w^2(s) - 2 w^3(s) + 2} \dif s
   .
\end{equation*}
%
Our goal is to show that $T$ has a fixed point in a subset of $C([0,
t_0])$.

\newpage

\textbf{Step 1.}
Let
%
\begin{equation}
   X = \cbr{w \in C([0, t_0]):  |w(t)| \leq 2 \text{ for all $t \in [0, t_0]$}}
   \label{eq:x}
\end{equation}
%
Show that if $w \in X$, then $Tw \in X$ if $t_0$ is small enough.

\textit{Solution.}

Fix $w \in X$. Then we have that
%
\begin{align*}
    \int_0^{t} \del{w^2(s) - 2 w^3(s) + 2} \dif s
        &\leq \int_0^{t} \envert{w^2(s) - 2 w^3(s) + 2} \dif s \\
        &\leq \int_0^{t} \del{|w^2(s)| + 2 |w^3(s)| + 2} \dif s \\
        &\leq \int_0^{t} \del{4 + 16 + 2} \dif s \\
        &= 22 t
        .
\end{align*}
%
Hence if we take then $t_0 \leq 1 / 22$ then
%
\begin{equation*}
    T w \leq 1 + 22 t \leq 1 + 22 t_0 \leq 2
\end{equation*}
%
and so $T w \in X$ for such $t_0$.

\newpage

\textbf{Step 2.}
Show that if $t_0$ is small enough then $T$ is a contraction mapping on
$(X, \norm{\cdot}_{L^\infty})$.

\textit{Solution.}

Let $w, v \in X$. We want to show that for $t_0$ sufficiently small,
then there exists a constant $k \in (0, 1)$ such that
%
\begin{equation*}
    \norm{Tw - Tv}_{L^\infty} \leq k \norm{w - v}_{L^\infty}
    .
\end{equation*}
%
The difference $Tw - Tv$ is trivially expressed by
%
\begin{equation*}
    Tw - Tv = \int_0^{t} \del{w^2(s) - v^2(s) - 2 w^3(s) + 2 v^3(s)} \dif s
    .
\end{equation*}
%
Note that in the integrand (suppressing the $s$ arguments for now) we can
rearrange the polynomial as follows
%
\begin{align*}
    w^2 - v^2 - 2 w^3 + 2 v^3
        &= (w - v) (w + v) - 2 (w - v) (w^2 + w v + v^2) \\
        &= (w - v) (w + v - 2 (w^2 + w v + v^2))
        .
\end{align*}
%
Using that, for any $t \in [0, t_0]$, $w - v \leq \norm{w -
v}_{L^\infty}$, we have the inequalities
%
\begin{align*}
    Tw - Tv
        &= \int_0^{t} (w(s) - v(s)) (w(s) + v(s) - 2 (w^2(s) + w(s) v(s) + v^2(s))) \dif s \\
        &\leq \int_0^{t} |w(s) - v(s)| |w(s) + v(s) - 2 (w^2(s) + w(s) v(s) + v^2(s))| \dif s \\
        &\leq \norm{w - v}_{L^\infty} \int_0^{t} |w(s) + v(s) - 2 (w^2(s) + w(s) v(s) + v^2(s))| \dif s \\
        &\leq \norm{w - v}_{L^\infty} \int_0^{t} \del{|w(s)| + |v(s)| + 2 (|w^2(s)| + |w(s)| |v(s)| + |v^2(s))|} \dif s \\
        &\leq \norm{w - v}_{L^\infty} \int_0^{t} \del{2 + 2 + 2 (4 + 4 + 4)} \dif s \\
        &\leq \norm{w - v}_{L^\infty} 28 t
        .
\end{align*}
%
Hence if we take $t_0 < 1 / 28$ then
%
\begin{equation*}
    \norm{Tw - Tv}_{L^\infty}
        \leq 28 \norm{w - v}_{L^\infty} \sup_{t \in [0, t_0} t
        \leq 28 \norm{w - v}_{L^\infty} t_0
        < \norm{w - v}_{L^\infty}
        .
\end{equation*}
%
Hence $T$ is a contraction mapping for such $t_0$.

\newpage

\textbf{Step 3.}
Show that $(X, \norm{\cdot}_{L^\infty})$ is complete, thus $T$ has a
unique fixed point in $X$.

\textit{Solution.}
Let $\cbr{w_n}$ be a Cauchy sequence in $X$. Define $w: [0, t_0] \to \R$
by $w(t) = \lim_{n \to \infty} w_n(t)$; that is, $w$ is the pointwise
limit of $\cbr{w_n}$. Note that this limit exists since $w_n(t)$ is a
Cauchy sequence in $\R$ (which is complete) for any fixed $t \in [0,
t_0]$.

Fix any $\epsilon > 0$. Since $\cbr{w_n}$ is Cauchy, there exists an $N
\in \N$ such that for all $n, m \geq N$,
%
\begin{equation*}
    \norm{w_n - w_m}_{L^\infty} < \frac{\epsilon}{3}
    .
\end{equation*}
%
First we will show that $w \in X$. To see that $|w(t)| \leq 2$ for all
$t \in [0, t_0]$. For any $t \in [0, t_0]$ we have that, using the
triangle inequality,
%
\begin{align*}
    |w(t)| &\leq |w(t) - w_N(t)| + |w_N(t)| \\
           &= \envert[2]{\lim_{n \to \infty} w_n(t) - w_N(t)} + |w_N(t)| \\
           &= \lim_{n \to \infty} \envert{w_n(t) - w_N(t)} + |w_N(t)| \\
           &< \frac{\epsilon}{3} + |w_N(t)| \\
           &\leq \frac{\epsilon}{3} + 2
           .
\end{align*}
%
Note that in the third step we were able to commute the limit with
$|\cdot|$ since $|\cdot|$ is continuous. Taking the limit as $\epsilon
\to 0$ we thus have that $|w(t)| \leq 2$ for all $t \in [0, t_0]$.

To see that $w$ is continuous, we will use the continuity of the $w_n$s:
there exists a $\delta > 0$ such that $d(t, s) < \delta$ implies that
%
\begin{equation*}
    |w_N(t) - w_N(s)| < \frac{\epsilon}{3}
    .
\end{equation*}
%
Hence, we have that, again using the triangle inequality,
%
\begin{align*}
    |w(t) - w(s)| &\leq |w(t) - w_N(t)| + |w(s) - w_N(s)| + |w_N(t) - w_N(s)| \\
                  &< 3 \frac{\epsilon}{3} \\
                  &= \epsilon
                  .
\end{align*}
%
This shows that $w$ is continuous. Hence, we have shown that $w \in X$.

Finally, we will show that $\cbr{w_n} \to w$. For all $t \in [0, t_0]$
we have that, for $n \geq N$,
%
\begin{align*}
    |w(t) - w_n(t)|
        &= \envert[2]{\lim_{m \to \infty} w_m(t) - w_n(t)} \\
        &= \lim_{m \to \infty} \envert{w_m(t) - w_n(t)} \\
        &\leq \frac{\epsilon}{3}
    .
\end{align*}
%
Hence for $n \geq N$,
%
\begin{equation*}
    \norm{w - w_m}_{L^\infty} < \frac{\epsilon}{3}
    .
\end{equation*}
%
which shows that $\cbr{w_n} \to w \in X$.

\newpage

\textbf{Step 4.}
Denote the fixed point in step 3 as $u$. Show that $u$ is not only
continuous but also differentiable on $[0, t_0]$ such that it is a
solution to~\eqref{eq:main}, by which we have found a unique solution
to~\eqref{eq:main}.

\textit{Solution.}
Since $u \in X$ it is automatically continuous. Further, since $u$ is
the unique fixed point of $T$ we have that
%
\begin{equation*}
    u = T u = 1 + \int_0^t \del{u^2(s) - 2 u^3(s) + 2} \dif s
\end{equation*}
%
Clearly $u$ is differentiable. Thus, taking a $t$ derivative on both
sides of the above equation, we obtain~\eqref{eq:main}. Hence $u$ is the
unique solution to~\eqref{eq:main}.

\newpage

\textbf{One more question.}
If we define a sequence of functions $u_n$ by
%
\begin{equation*}
   u_0 = 1, \qquad u_{n+1} = Tu_n \text{ for all $n \geq 0$}
   ,
\end{equation*}
%
show that $\cbr{u_n}_{n = 0}^\infty$ must have a convergence subsequence
in $(X, \norm{\cdot}_{L^\infty})$, where $X$ is defined
in~\eqref{eq:x} with the same $t_0$ chosen in the previous steps.

\textit{Solution.}
To keep the notation simple in this solution, let $d =
\norm{\cdot}_{L^\infty}$. Our goal is to show that $\cbr{u_n}$ is
Cauchy. Since we have shown that $X$ is complete, it then follows
automatically that it has a convergent subsequence.

First, note that
%
\begin{equation*}
    u_1 - u_0 = T u_0 -  u_0 = \del{1 + \int_0^t \del{1 - 2 + 2} \dif s} - 1 = t
\end{equation*}
%
and hence $d(u_1, u_0) = t_0$.

Fix any $\epsilon > 0$. Since $T$ is a contraction mapping, there exists
$k \in (0, 1)$ such that $d(Tw, Tv) \leq k d(w, v)$ for all $w, v \in
X$. In particular we have that
%
\begin{equation*}
    d(u_{n + 1}, u_n) = d(T u_n, T u_{n - 1}) \leq k d(u_n, u_{n - 1}) \leq k^2 d(u_{n - 1}, u_{n - 2}) \leq \cdots \leq k^n d(u_1, u_0) = k^n t_0
    .
\end{equation*}
%
Let $n, m \geq N$, without loss of generality taking $n \geq m$, where $N$ is sufficiently large such that
%
\begin{equation*}
    t_0 \frac{k^N}{1 - k} < \epsilon
    .
\end{equation*}
%
Then by repeatedly applying the triangle equality we have
%
\begin{align*}
    d(u_n, u_m) &\leq d(u_n, u_{n - 1}) + d(u_{n - 1}, u_{n - 2}) + \cdots + d(u_{m + 1}, u_m) \\
                &\leq k^{n - 1} t_0 + k^{n - 2} t_0 + \cdots + k^{m} t_0 \\
                &\leq t_0 (k^m + k^{m + 1} + k^{m + 2} + \cdots) \\
                &= t_0 \frac{k^m}{1 - k} \\
                &\leq t_0 \frac{k^N}{1 - k} \\
                &< \epsilon
                .
\end{align*}
%
Hence we have shown that $\cbr{u_n}$ is Cauchy and the completeness of
$X$ implies it has a convergent subsequence.

\end{document}
