% Set up the document
\documentclass{article}

% Page size
\usepackage[
    letterpaper,]{geometry}

% Lines between paragraphs
\setlength{\parskip}{\baselineskip}
\setlength{\parindent}{0pt}

% Math
\usepackage{mathtools}
\usepackage{amssymb}
\usepackage{amsthm}
\usepackage{commath}

% Number sets
\newcommand{\C}{\mathcal{C}}
\newcommand{\N}{\mathbb{N}}
\newcommand{\Q}{\mathbb{Q}}
\newcommand{\R}{\mathbb{R}}
\newcommand{\Z}{\mathbb{Z}}

% Links
\usepackage{hyperref}

% Page numbers at top right
\usepackage{fancyhdr}
\pagestyle{fancy}
\fancyhf{}
\fancyhead[R]{\thepage}
\renewcommand\headrulewidth{0pt}

\begin{document}

\textbf{MATH 320 Homework 9} \\
\textbf{Matt Wiens \#301294492} \\
\textbf{2020-04-14}

Consider the equation
%
\begin{equation}
    \dod{u}{t} = u^2 - 2 u^3 + 2, \qquad \eval[1]{u}_{t = 0} = 1
    .
    \label{eq:main}
\end{equation}
%
Show that there exists $t_0 > 0$ such that~\eqref{eq:main}
has a unique solution $u(t)$ for $t \in [0, t_0]$.

\vspace{5mm}

\textbf{Main setting.} Convert~\eqref{eq:main} into its integral form by
integrating both sides in $t$:
%
\begin{align*}
    u(t) &= u(0) + \int_0^t \del{u^2(s) - 2 u^3(s) + 2} \dif s \\
         & = 1 + \int_0^t \del{u^2(s) - 2 u^3(s) + 2} \dif s
         .
\end{align*}
%
Define the operator $T: C([0, t_0]) \to C([0, t_0])$ (with $t_0$ to be
determined) as
%
\begin{equation*}
   Tw = 1 + \int_0^t \del{w^2(s) - 2 w^3(s) + 2} \dif s
   .
\end{equation*}
%
Our goal is to show that $T$ has a fixed point in a subset of $C([0,
t_0])$.

\newpage

\textbf{Step 1.}
Let
%
\begin{equation}
   X = \cbr{w \in C([0, t_0]):  |w(t)| \leq 2 \text{ for all $t \in [0, t_0]$}}
   \label{eq:x}
\end{equation}
%
Show that if $w \in X$, then $Tw \in X$ if $t_0$ is small enough.

\textit{Solution.}

\newpage

\textbf{Step 2.}
Show that if $t_0$ is small enough then $T$ is a contraction mapping on
$(X, \norm{\cdot}_{L^\infty})$.

\textit{Solution.}

\newpage

\textbf{Step 3.}
Show that $(X, \norm{\cdot}_{L^\infty})$ is complete, thus $T$ has a
unique fixed point in $X$.

\textit{Solution.}

\newpage

\textbf{Step 4.}
Denote the fixed point in step 3 as $u$. Show that $u$ is not only
continuous but also differentiable on $[0, t_0]$ such that it is a
solution to~\eqref{eq:main}, by which we have found a unique solution
to~\eqref{eq:main}.

\textit{Solution.}

\newpage

\textbf{One more question.}
If we define a sequence of functions $u_n$ by
%
\begin{equation*}
   u_0 = 1, \qquad u_{n+1} = Tu_n \text{ for all $n \geq 0$}
   ,
\end{equation*}
%
show that $\cbr{u_n}_{n = 0}^\infty$ must have a convergence subsequence
in $(X, \norm{\cdot}_{L^\infty})$, where $X$ is defined
in~\eqref{eq:x} with the same $t_0$ chosen in the previous steps.

\textit{Solution.}

\end{document}
