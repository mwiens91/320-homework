% Set up the document
\documentclass{article}

% Page size
\usepackage[
    letterpaper,]{geometry}

% Lines between paragraphs
\setlength{\parskip}{\baselineskip}
\setlength{\parindent}{0pt}

% Math
\usepackage{mathtools}
\usepackage{amssymb}
\usepackage{amsthm}
\usepackage{commath}

% Number sets
\newcommand{\C}{\mathcal{C}}
\newcommand{\N}{\mathbb{N}}
\newcommand{\Q}{\mathbb{Q}}
\newcommand{\R}{\mathbb{R}}
\newcommand{\Z}{\mathbb{Z}}

% Links
\usepackage{hyperref}

% Page numbers at top right
\usepackage{fancyhdr}
\pagestyle{fancy}
\fancyhf{}
\fancyhead[R]{\thepage}
\renewcommand\headrulewidth{0pt}

\begin{document}

\textbf{MATH 320 Worksheet 2} \\
\textbf{Matt Wiens \#301294492} \\
\textbf{2020-01-23}

1. Let $E$ be the set of all $x \in [0, 1]$ whose decimal expansion
   contains only the digits $4$ and $7$. Is $E$ countable? Is $E$ dense
   in $[0, 1]$? Is $E$ compact? Is $E$ perfect?

\textit{Solution.}
$E$ is uncountable. To prove this, we will show that for any countable
subset $X \subseteq E$, $X \neq E$. Let $X$ be a countable subset of $E$
and represent $X$ as a (possibly infinite) sequence $X = \cbr{x_1, x_2,
\ldots}$. We will show that there exists an element $y \in E$ such that
$y \not\in X$. Let the decimal expansion of $y$ be
%
\begin{equation*}
    y = 0 . d_1 d_2 \ldots
    ,
\end{equation*}
%
where for each $i = 1, 2, \ldots$, $d_i = 4$ if the $i$th decimal
position of $x_i$ is $7$, or $d_i = 7$ if the $i$th decimal position of
$x_i$ is $4$. Clearly $y \in E$, but since $y$ disagrees with each $x_i$
in the $i$th decimal position, $y \not\in X$. Thus $X \neq E$, and since
$X$ was an arbitrary countable subset of $E$, it follows that $E$ is not
countable.

$E$ is not dense in $[0, 1]$. To show why, consider the point $0.9 \in
[0, 1]$. Here we have that $N_{0.01}(0.9) \cap E = \emptyset$, so $0.9$
is neither in $E$ nor a limit point of $E$.

$E$ is compact. Here we will use the Heine-Borel theorem. Clearly $E$ is
bounded. To show that $E$ is closed, let $y \in [0, 1] \setminus E$.
Since $y \not\in E$, at some $i$th decimal position, the $i$th decimal
position of $y$ is neither $4$ nor $7$. Let $\alpha = \frac{1}{10^{i +
1}}$. Then $N_\alpha (y) \cap E = \emptyset$, so $y$ is not a limit
point of $E$. Thus $E$ is closed. Since $E$ is closed and bounded, it is
compact.

$E$ is perfect. We have already shown that $E$ is closed, so we just
need to show that all points $x \in E$ are limit points of $E$. Consider
any $x \in E$ and $r > 0$. Let $\alpha = \frac{1}{10^{i}}$ where $i$ is
chosen such that $r > \alpha$. Consider $y \in E$ where the decimal
expansion of $y$ is identical to $x$ except that the $(i + 1)$th digit
of $y$ is $4$ is the $(i + 1)$th digit of $x$ is $7$, or the $(i + 1)$th
digit of $y$ is $7$ if the $(i + 1)$th digit of $x$ is $4$. Then
$\envert{x - y} = \frac{3}{10^{i + 1}} < \frac{1}{10^i} = \alpha$. Hence
$y \in N_\alpha (x) \subset N_r(x)$. Since $r$ was arbitrary, it follows
that $x$ is a limit point of $E$. Since $x$ was arbitrary, it follows
that all points $x \in E$ are limit points of $E$. Thus, given that
we've already shown that $E$ is closed, $E$ is perfect.

\newpage

2. Consider the metric space $\del{C([0, 1]), d}$ of continuous real-valued
   functions on the interval $[0, 1]$ with the sup-norm metric
%
\begin{equation*}
    d(f, g) \coloneqq \sup_{t \in [0, 1]} \envert{f(t) - g(t)}
    .
\end{equation*}
%
Consider the subset $A$ given by
%
\begin{equation*}
    A \coloneqq \cbr{f \in C([0, 1]): |f(t)| \leq 5 \quad \forall t \in [0, 1]}
    .
\end{equation*}
%
Is this subset open? Closed? Compact?

\textit{Solution.}
$A$ is not open. Consider the constant-valued function $f: x \mapsto 5$.
Clearly $f \in A$. However, for any $r > 0$, given the function $g: x
\mapsto 5 + \frac{r}{2}$, we have that $g \in N_r(f)$ and $g \not\in A$.
Thus $f$ is not an interior point, so $A$ is not open.

$A$ is closed. To prove this, we will show that $A^c$ is open. Consider
any $f \in A^c$. Let $t_c \in [0, 1]$ be the point satisfying $\min_{t
\in [0, 1]} \envert{\,\envert{f(t)} - 5} = \envert{\,\envert{f(t_c)} -
5} = \alpha$. Note that the continuity of $f$ guarantees that such a
$t_c$ exists. Then $N_\alpha (f) \subset A^c$, and hence $f$ is an
interior point of $A^c$. Since $f$ was arbitrary, it follows that $A^c$
is open, and hence $A$ is closed.

$A$ is compact, although I'm not sure how to prove that it is.

\newpage

3. Consider the metric space $(X, d)$ where $d$ is the discrete metric.
   Show that every subset of $X$ is both open and closed. Show that if
   $X = \R$, then the set $\cbr{x \in \R: 0 \leq x \leq 1}$ is not
   compact.

\textit{Solution.}
Let $S \subseteq X$. First we will show that $S$ is open. Consider any
$x \in S$. Then for any radius $r$ satisfying $0 < r < 1$, $N_r(x) =
\cbr{x} \subseteq S$, so $x$ is an interior point of $S$.

To show that $S$ is closed, consider that all punctured neighbourhoods
with radii $r$ satisfying $0 < r < 1$ for all points in $X$ are empty,
so $S$ cannot have any limit points. Hence $S$ is closed.

$A$ is not compact. To see why, consider the open cover
$\cbr{\cbr{x}}_{x \in [0, 1]}$. Clearly
%
\begin{equation*}
    [0, 1] \subseteq \bigcup_{x \in [0, 1]} \cbr{x}
    ,
\end{equation*}
%
but this cover does not admit any finite subcover.

\end{document}
