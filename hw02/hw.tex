% Set up the document
\documentclass{article}

% Page size
\usepackage[
    letterpaper,]{geometry}

% Lines between paragraphs
\setlength{\parskip}{\baselineskip}
\setlength{\parindent}{0pt}

% Math
\usepackage{mathtools}
\usepackage{amssymb}
\usepackage{amsthm}
\usepackage{commath}

% Number sets
\newcommand{\C}{\mathcal{C}}
\newcommand{\N}{\mathbb{N}}
\newcommand{\Q}{\mathbb{Q}}
\newcommand{\R}{\mathbb{R}}
\newcommand{\Z}{\mathbb{Z}}

% Links
\usepackage{hyperref}

% Page numbers at top right
\usepackage{fancyhdr}
\pagestyle{fancy}
\fancyhf{}
\fancyhead[R]{\thepage}
\renewcommand\headrulewidth{0pt}

\begin{document}

\textbf{MATH 320 Homework 2} \\
\textbf{Matt Wiens \#301294492} \\
\textbf{2020-01-24}

1. Consider the metric space $(M, d)$ where $M$ is all $n \times n$
matrices of real numbers, with the distance $d(A, B)$ between $A, B
\in M$ defined as $d(A, B)= \sqrt{\sum_{i, j = 1}^n |A_{i j} - B_{i j}|^2}$.

Show that the set of orthogonal $n \times n$ matrices in this metric
space is a compact set. Recall that a matrix is orthogonal if and only
if $M^t M = I$.

\begin{proof}

Note that we view the space $M$ as $\R^{n^2}$, where the norm $d$ is
simply the $l^2$ norm. We can do this through a bijection $f: M \to
\R^{n^2}$ which takes in a matrix $A \in M$ and returns an element $v
\in \R^{n^2}$ where each matrix entry $A_{ij}$ is the same as the entry
$v_{n (i - 1) + j}$. Hence we can use Heine Borel here.

Let $E \subset M$ denote the set of orthogonal $n \times n$ matrices.

To show that $E$ is bounded, consider any $A \in E$. Since $A^t A = I$,
we have that for each row $i = 1, \ldots, n$,
%
\begin{equation*}
    \sum_{j = 1}^n A_{ij}^2 = 1
    .
\end{equation*}
%
Summing over all rows, we have
%
\begin{equation*}
    \sum_{i, j = 1}^n A_{ij}^2 = n
    .
\end{equation*}
%
Therefore, if we denote $0 \in M$ as the matrix whose entries are all
$0$, we have that for all $A \in E$,
%
\begin{align*}
    d(A, 0)
        &= \sqrt{\sum_{i, j = 1}^n |A_{i j} - 0_{i j}|^2} \\
        &= \sqrt{\sum_{i, j = 1}^n |A_{i j}|^2} \\
        &= \sqrt{\sum_{i, j = 1}^n A_{i j}^2} \\
        &= \sqrt{n} \\
        &< \sqrt{n} + 1
        .
\end{align*}
%
Thus $E$ is bounded.

\textbf{need to show closed}

\end{proof}

\newpage

2. Consider the set $X$ of real sequences of the form $\cbr{a_n}_{ n=
   1}^\infty$, along with the function $d: X \times X \rightarrow [0,
   +\infty)$ defined as
%
\begin{equation*}
    d(x, y) \coloneqq \sum_{n = 1}^\infty |x_n - y_n|,
    \qquad \forall x = \cbr{x_n}, y = \cbr{y_n} \in X
    .
\end{equation*}
%
(a) Is $(X,d)$ a metric space? Prove/disprove.

\begin{proof}

We need to show that $d$ is a metric of $X$. Let $x, y, z \in X$.

Clearly $d(x, y) \geq 0$. Additionally,
%
\begin{align*}
    &d(x, y) = 0 \\
    &\iff \envert{x_n - y_n} = 0, \quad \forall n = 1, 2, \ldots \\
    &\iff x_n = y_n, \quad \forall n = 1, 2, \ldots \\
    &\iff x = y
    .
\end{align*}
%
Hence $d$ satisfies the first property of a metric.

To show that $d$ is symmetric, consider that
%
\begin{align*}
    d(x, y)
        &= \sum_{n = 1}^\infty |x_n - y_n| \\
        &= \sum_{n = 1}^\infty |y_n - x_n| \\
        &= d(y, x)
        .
\end{align*}
%
Now we only need to show that $d$ satisfies the triangle inequality.
Here we will use the triangle inequality of $|\cdot|$:
%
\begin{align*}
    d(x, y) + d(y, z)
        &= \sum_{n = 1}^\infty |x_n - y_n| + \sum_{n = 1}^\infty |y_n - z_n| \\
        &= \sum_{n = 1}^\infty \del{|x_n - y_n| + |y_n - z_n|} \\
        &\geq \sum_{n = 1}^\infty |x_n - z_n| \\
        &= d(x, z)
        .
\end{align*}
%
Therefore $d$ is a metric on $X$ and $(X, d)$ is a metric space.

\end{proof}

(b) Is the subset $A$ of sequences $\cbr{a_n}$ such that $\sum_{n =
1}^\infty |a_n| \geq 1$ open in $X$? Compact? Prove your assertions.

\textit{Solution.}
$A$ is not open. Consider the sequence $a = \cbr{a_n}$ where
%
\begin{equation*}
    a_n = \begin{cases}
        1, &n = 1 \\
        0, &n > 1.
    \end{cases}
\end{equation*}
%
Clearly $a \in A$. Now, fix any $r > 0$, and let $\alpha =
\max\cbr{1, r}$. Consider the sequence $b = \cbr{b_n}$ where
%
\begin{equation*}
    b_n = \begin{cases}
        1 - \frac{\alpha}{2}, &n = 1 \\
        0, &n > 1.
    \end{cases}
\end{equation*}
%
Now, $\sum_{n = 1}^\infty |b_n| = 1 - \frac{\alpha}{2} < 1$, so $b
\not\in A$. However $d(a, b) = \frac{\alpha}{2} < r$. Therefore $N_r(a)
\not\subset A$ for all $r > 0$. Therefore $a$ is not an interior point
and $A$ is not open.

\textbf{need compactness}

\newpage

3. Consider the metric space $(X, d)$ where $d$ is the discrete metric.
   Show that every subset of $X$ is both open and closed. Show that if
   $X = \R$, then the set $\cbr{x \in \R: 0 \leq x \leq 1}$ is not
   compact.

\begin{proof}

Let $S \subseteq X$. First we will show that $S$ is open. Consider any
$x \in S$. Then for any radius $r$ satisfying $0 < r < 1$, $N_r(x) =
\cbr{x} \subseteq S$, so $x$ is an interior point of $S$.

To show that $S$ is closed, consider that all punctured neighbourhoods
with radii $r$ satisfying $0 < r < 1$ for all points in $X$ are empty,
so $S$ cannot have any limit points. Hence $S$ is closed.

$A$ is not compact. To see why, consider the open cover
$\cbr{\cbr{x}}_{x \in [0, 1]}$. Clearly
%
\begin{equation*}
    [0, 1] \subseteq \bigcup_{x \in [0, 1]} \cbr{x}
    ,
\end{equation*}
%
but this cover does not admit any finite subcover.

\end{proof}

\end{document}
