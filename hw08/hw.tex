% Set up the document
\documentclass{article}

% Page size
\usepackage[
    letterpaper,]{geometry}

% Lines between paragraphs
\setlength{\parskip}{\baselineskip}
\setlength{\parindent}{0pt}

% Math
\usepackage{mathtools}
\usepackage{amssymb}
\usepackage{amsthm}
\usepackage{commath}

% Number sets
\newcommand{\C}{\mathcal{C}}
\newcommand{\N}{\mathbb{N}}
\newcommand{\Q}{\mathbb{Q}}
\newcommand{\R}{\mathbb{R}}
\newcommand{\Z}{\mathbb{Z}}

% Links
\usepackage{hyperref}

% Page numbers at top right
\usepackage{fancyhdr}
\pagestyle{fancy}
\fancyhf{}
\fancyhead[R]{\thepage}
\renewcommand\headrulewidth{0pt}

\begin{document}

\textbf{MATH 320 Homework 8} \\
\textbf{Matt Wiens \#301294492} \\
\textbf{2020-04-14}

1. Let $f: \R^2 \to \R^2$ and $g: \R^2 \to \R^3$ be defined by
%
\begin{align*}
    f(x, y) &= (x \sin y, \ \cos(x+y)), \\
    g(x, y) &= (x^2 y, \ \ln (1+(x+y)^2))
    .
\end{align*}
%
Use the chain rule (instead of finding the explicit formula of $g \circ
f$) to find $D(f \circ g)$ at $(1, 0)$.

\textit{Solution.}
There's a typo in this question.

\newpage

2. Let $f: \R^k \to \R^n$ and $g: \R^k \to \R^n$ be differentiable
mappings. Use the definition of the derivative to show that $\alpha f
+ \beta g: \R^k \to \R^n$ is differentiable for any $\alpha, \beta
\in \R$.

\textit{Solution.}

\newpage

3. Consider the function $f: \R^2 \to \R$ given by
%
\begin{equation*}
    f(x, y)
    = \begin{cases}
        \dfrac{x y}{x^2 + y^2}, & (x, y) \neq (0, 0), \\
        0, & (x, y) = (0, 0).
   \end{cases}
\end{equation*}
%
Show that $\partial x f$ and $\partial y f$ both exist at $(0, 0)$ but
$f$ is not differentiable at $(0, 0)$, hence $D f$ does not exist at
$(0, 0)$.

\textit{Solution.}

\newpage

4. Consider the two equations
%
\begin{align*}
    x^2 - y^2 - u^3 + v^2 + 4 = 0
    \quad \text{and} \quad
    2 x y + y^2 - 2 u^2 + 3 v^4 + 8 = 0
    .
\end{align*}
%
Show that near the point $(x, y, u, v) = (2, -1, 2, 1)$, we can solve
for $(u, v)$ in terms of $(x, y)$.

\textit{Solution.}

\end{document}
