% Set up the document
\documentclass{article}

% Page size
\usepackage[
    letterpaper,]{geometry}

% Lines between paragraphs
\setlength{\parskip}{\baselineskip}
\setlength{\parindent}{0pt}

% Math
\usepackage{mathtools}
\usepackage{amssymb}
\usepackage{amsthm}
\usepackage{commath}

% Operators
\DeclareMathOperator{\sgn}{sgn}

% Number sets
\newcommand{\C}{\mathcal{C}}
\newcommand{\N}{\mathbb{N}}
\newcommand{\Q}{\mathbb{Q}}
\newcommand{\R}{\mathbb{R}}
\newcommand{\Z}{\mathbb{Z}}

% Links
\usepackage{hyperref}

% Page numbers at top right
\usepackage{fancyhdr}
\pagestyle{fancy}
\fancyhf{}
\fancyhead[R]{\thepage}
\renewcommand\headrulewidth{0pt}

\begin{document}

\textbf{MATH 320 Homework 7} \\
\textbf{Matt Wiens \#301294492} \\
\textbf{2020-03-20}

1. Consider the sequence of functions $\cbr{f_n}_{ n =1}^\infty$ given
by
%
\begin{equation*}
    f_n(x) = \frac{1}{1 + n x},
    \qquad
    x \in [0, 1]
    .
\end{equation*}
%
(a) Prove that every $f_n$ is uniformly bounded and uniformly continuous.

\begin{proof}

Note that for all $n \in \N$,
%
\begin{equation*}
    \sup_{x \in [0, 1]} |f_n(x)|
    = \sup_{x \in [0, 1]} f_n(x)
    = f_n(0)
    = 1
    .
\end{equation*}
%
Hence $\cbr{f_n}_{n = 1}^\infty$ is uniformly bounded.

Recall from lectures that if a function $f$ is continuous on a compact
set $X$, then $f$ is uniformly continuous on that set. Since $[0, 1]$ is
compact, to show that each $f_n$ is uniformly continuous on $[0, 1]$ we
need only to show that they are continuous on the set. Here, we will show
that each $f_n$ is continuous on every point on $[0, 1]$, which implies
that it is continuous everywhere on the interval.

Fix any $x_0 \in [0, 1]$ and $\epsilon > 0$. Then if $|x - x_0| <
\frac{\epsilon}{n}$, we have that
%
\begin{align*}
    |f_n(x) - f_n(x_0)|
        &= \envert{\frac{1}{1 + n x} - \frac{1}{1 + n x_0}} \\
        &= \frac{n |x_0 - x|}{\envert{n^2 x x_0 + n x + n x_0 + 1}} \\
        &\leq n |x_0 - x| \\
        &< n \, \frac{\epsilon}{n} \\
        &= \epsilon
        ,
\end{align*}
%
where in the third step we used that $n, x, x_0 \geq 0$ implies that
%
\begin{equation*}
    \envert{n^2 x x_0 + n x + n x_0 + 1}
    = n^2 x x_0 + n x + n x_0 + 1
    \geq 1
    .
\end{equation*}

\end{proof}

(b) Prove by definition that $\cbr{f_n}_{n = 1}^\infty$ is not
equicontinuous.

\begin{proof}

Suppose for contradiction that $\cbr{f_n}_{n = 1}^\infty$ was
equicontinuous. Let $\epsilon = \frac{1}{2}$. Then there would exist
$\delta > 0$ such that for all $x \in [0, 1]$ satisfying $|x - 0| = x <
\delta$, $|f_n(x) - f_n(0)| < \epsilon$ for all $n \in N$. However,
if we take $n > \frac{1}{x}$ then $n x > 1$ and
%
\begin{align*}
    |f_n(x) - f_n(0)|
        &= \envert{\frac{1}{1 + n x} - 1} \\
        &= \envert{\frac{n x}{1 + n x}} \\
        &= \frac{n x}{1 + n x} \\
        &> \frac{n x}{n x + n x} \\
        &> \frac{1}{2} \\
        &= \epsilon
        .
\end{align*}
%
This contradicts our assumption that $\cbr{f_n}_{n = 1}^\infty$ was
equicontinuous.

\end{proof}

\newpage

2. Let
%
\begin{equation*}
    F_n(x) = \int_\pi^x n \sin \del{\frac{e^z}{n}} \dif z,
    \qquad
    x \in [\pi, 2 \pi]
    .
\end{equation*}
%
Prove that there exists a subsequence $\cbr{F_{n_k}}_{k = 1}^\infty$
which converges uniformly on $[\pi, 2 \pi]$.

\begin{proof}

First, note that $\cbr{F_n(x)}_{n = 1}^\infty \subset C[\pi, 2 \pi]$ and
that $[\pi, 2 \pi]$ is compact in $\R$. Then, according to Ascali-Arzela,
$\cbr{F_n(x)}_{n = 1}^\infty$ will have a uniformly convergent
subsequence $\cbr{F_{n_k}}_{k = 1}^\infty$ provided that
$\cbr{F_n(x)}_{n = 1}^\infty$ is uniformly bounded and equicontinuous.

To show uniform boundedness note that we can write the integrand in the form
%
\begin{equation*}
    n \sin \del{\frac{e^z}{n}}
    = \frac{\sin \del{\frac{1}{n} e^z}}{\frac{1}{n} e^z} e^z
    ,
\end{equation*}
%
Recall from calculus that sinc function $\frac{sin(x)}{x}$ achieves its
supremum at zero with a value of $1$. Hence
%
\begin{equation*}
    n \sin \del{\frac{e^z}{n}}
        \leq \frac{\sin \del{\frac{1}{n} e^z}}{\frac{1}{n} e^z} e^{2 \pi}
        \leq e^{2 \pi}
    .
\end{equation*}
%
Therefore for all $n \in N$ we have that
\begin{equation*}
    F_n(x) = \int_\pi^x n \sin \del{\frac{e^z}{n}} \dif z
        \leq \int_\pi^x e^{2 \pi} \dif z
        = (x - \pi) e^{2 \pi}
        \leq \pi e^{2 \pi}
        .
\end{equation*}

To show equicontinuity, fix any $\epsilon > 0$ and let $\delta =
\frac{\epsilon}{e^{2 \pi}}$. Then we have that if $|x - y| < \delta$
(without loss of generality taking $x \geq y$), for all $n \in \N$
%
\begin{align*}
    |f_n(x) - f_n(y)|
        &= \envert{\int_y^x n \sin \del{\frac{e^z}{n}} \dif z} \\
        &= \int_y^x n \sin \del{\frac{e^z}{n}} \dif z \\
        &\leq \int_y^x e^{2 \pi} \dif z \\
        &= e^{2 \pi} |x - y| \\
        &< e^{2 \pi} \frac{\epsilon}{e^{2 \pi}} \\
        &= \epsilon
        .
\end{align*}

\end{proof}

\newpage

3. Let $f: [0, 1] \to \R$ be defined as
%
\begin{equation*}
    f(x)
    = \begin{cases}
        1, & x = 1/n \ \text{for any $n \in \N$} \\
        \sin x, & \text{otherwise}
    \end{cases}
    .
\end{equation*}
%
(a) Show that for any $\epsilon > 0$, there exists a continuous function
$g$ defined on $[0, 1]$ such that
%
\begin{equation*}
    \int_0^1 |f(x) - g(x)|^2 \dif x < \epsilon
    .
\end{equation*}

\begin{proof}

Let $g(x) = \sin(x)$. Fix any $\epsilon > 0$. Then define the function
%
\begin{equation*}
    F(x) =
    \begin{cases}
        1 - 2 \sin(x) + \sin^2(x), & x = 1/n \ \text{for any $n \in \N$} \\
        0, & \text{otherwise}
    \end{cases}
    ,
\end{equation*}
%
and also
%
\begin{equation*}
    G(x) =
    \begin{cases}
        1, & x = 1/n \ \text{for any $n \in \N$} \\
        0, & \text{otherwise}
    \end{cases}
    .
\end{equation*}
%
Note that $|F(x)| \leq 4 G(x)$. Then we have that
%
\begin{align*}
    \int_0^1 |f(x) - g(x)|^2 \dif x
        = \int_0^1 |F(x)| \dif x
        \leq 4 \int_0^1 G(x) \dif x
        = 0
        < \epsilon
        .
\end{align*}
%
Where we used the fact from introductory analysis that
%
\begin{equation*}
    \int_0^1 G(x) \dif x = 0
    .
\end{equation*}

\end{proof}

(b) Show that for any $\epsilon > 0$, there exists a polynomial $h$ such that
%
\begin{equation*}
    \int_0^1 |f(x) - h(x)|^2 \dif x < \epsilon
    .
\end{equation*}

\begin{proof}

Fix any $\epsilon > 0$. Let $g \in C[0, 1]$ be such that
%
\begin{equation*}
    \int_0^1 |f(x) - g(x)|^2 \dif x < \frac{\epsilon}{2}
    ,
\end{equation*}
%
and, using that the set of polynomials are dense in $(C[0, 1],
d_\infty)$, let $h$ be a polynomial such that $d_\infty(g, h) <
\sqrt{\frac{\epsilon}{2}}$. Then we have that
%
\begin{equation*}
    \int_0^1 |g(x) - h(x)|^2 \dif x
        < \int_0^1 \del{\frac{\epsilon}{2}}^2 \dif x
        = \frac{\epsilon}{2}
        ;
\end{equation*}
%
and so
%
\begin{align*}
    \int_0^1 |f(x) - h(x)|^2 \dif x
        &\leq \int_0^1 \del{|f(x) - g(x)|^2 + |g(x) - h(x)|^2} \dif x \\
        &= \int_0^1 |f(x) - g(x)|^2 \dif x + \int_0^1 |g(x) - h(x)|^2 \dif x \\
        &< \frac{\epsilon}{2} + \frac{\epsilon}{2} \\
        &= \epsilon
        .
\end{align*}

\end{proof}

\newpage

4. Put $P_0 = 0$, and define, for $n = 0, 1, 2, \ldots,$
%
\begin{equation*}
    P_{n + 1}(x) = P_n(x) + \frac{1}{2} \del{x^2 - P_n^2(x)}
    .
\end{equation*}
%
Prove that
%
\begin{equation*}
    \lim_{n \to \infty} P_n(x) = |x|
    ,
\end{equation*}
%
uniformly on $[-1, 1]$.

\begin{proof}

We will use the following identity suggested in the textbook:
%
\begin{equation}
    |x| - P_{n + 1}(x) = \del{|x| - P_n(x)} \del{1 - \frac{|x| + P_n(x)}{2}}
    \label{eq:4-id}
    .
\end{equation}
%
Furthermore, we will restrict all values of $x$ below to lie in $[-1, 1]$.

First let's show that $0 \leq P_n(x) \leq |x|$. For $n = 1$, clearly
%
\begin{equation*}
    P_1(x) = \frac{x^2}{2} \leq |x|
\end{equation*}
%
and note also that the above equality implies $P_1(x) \geq 0$. Suppose
this holds for $n = k - 1$. Then we have
%
\begin{equation*}
    P_k(x) = P_{k - 1}(x) + \frac{1}{2} \del{x^2 - P^2_{k - 1}(x)}
    .
\end{equation*}
%
By assumption $P_{k - 1}(x) \geq 0$ and also $\del{x^2 - P^2_{k - 1}(x)}
\geq 0$ and hence $P_k \geq 0$. Using~\eqref{eq:4-id} we have
%
\begin{equation*}
    |x| - P_{k}(x) = \del{|x| - P_{k - 1}(x)} \del{1 - \frac{|x| + P_{k - 1}(x)}{2}} \geq 0
\end{equation*}
%
since
%
\begin{equation*}
    |x| - P_{k - 1}(x) \geq 0
\end{equation*}
%
and
%
\begin{equation*}
    1 - \frac{|x| + P_{k - 1}(x)}{2} \geq 0
\end{equation*}
%
(the last inequality is true since $0 \leq P_{k - 1} \leq |x| \leq 1$).
This completes the induction, and hence we have that $0 \leq P_n(x) \leq
|x|$ for all $n \in \N$. That $P_{n + 1}(x) \geq P_n$ follows trivially
now from~\eqref{eq:4-id}. Hence we have that $0 \leq P_n(x) \leq P_{n +
1}(x) \leq |x|$.

Observe that
%
\begin{equation*}
    |x| - P_{n}(x)
        = \del{|x| - P_{n - 1}(x)} \del{1 - \frac{|x| + P_{n - 1}(x)}{2}}
        \leq \del{|x| - P_{n - 1}(x)} \del{1 - \frac{|x|}{2}}
\end{equation*}
%
and so
%
\begin{align*}
    0
        &\leq |x| - P_n(x) \\
        &\leq \del{|x| - P_{n - 1}(x)} \del{1 - \frac{|x|}{2}} \\
        &\leq \del{|x| - P_{n - 2}(x)} \del{1 - \frac{|x|}{2}}^2 \\
        &\leq \ldots \\
        &\leq (|x| - 0) \del{1 - \frac{|x|}{2}}^n \\
        &= |x| \del{1 - \frac{|x|}{2}}^n
        .
\end{align*}
%
Noting that for $y \in [0, 1]$ we have
%
\begin{align*}
    y \del{1 - \frac{y}{2}}^n
        &\leq y (1 - y)^n \\
        &\leq y^n (1 - y)^n \\
        &= \del{y (1 - y)}^n \\
        &= \del{\frac{1}{4} - \del{\frac{1}{2} - y}^2}^n \\
        &\leq \frac{1}{4^n}
        ,
\end{align*}
%
it follows that
%
\begin{equation*}
    0 \leq |x| - P_n(x) \leq \frac{1}{4^n}
    .
\end{equation*}
%
Hence $\lim_{n \to \infty} P_n(x) = |x|$ uniformly on $[-1, 1]$.

\end{proof}


\end{document}
