% Set up the document
\documentclass{article}

% Page size
\usepackage[
    letterpaper,]{geometry}

% Lines between paragraphs
\setlength{\parskip}{\baselineskip}
\setlength{\parindent}{0pt}

% Math
\usepackage{mathtools}
\usepackage{amssymb}
\usepackage{amsthm}
\usepackage{commath}

% Number sets
\newcommand{\C}{\mathcal{C}}
\newcommand{\N}{\mathbb{N}}
\newcommand{\Q}{\mathbb{Q}}
\newcommand{\R}{\mathbb{R}}
\newcommand{\Z}{\mathbb{Z}}

% Links
\usepackage{hyperref}

% Page numbers at top right
\usepackage{fancyhdr}
\pagestyle{fancy}
\fancyhf{}
\fancyhead[R]{\thepage}
\renewcommand\headrulewidth{0pt}

\begin{document}

\textbf{MATH 320 Homework 4} \\
\textbf{Matt Wiens \#301294492} \\
\textbf{2020-02-07}

1. Suppose $\cbr{x_n}$ is a Cauchy sequence in $(\R, d_2)$ where $d_2$ is
the usual Euclidean metric. Show that the sequence $\cbr{f(x_n)}$ is
convergent, where the function $f$ is defined by $f(x) \coloneqq x^2$.

\begin{proof}

Fix any $\epsilon > 0$. Since $(\R, d_2)$ is complete, $\cbr{x_n} \to x$
for some $x \in \R$. Therefore, there exists $M \in \N$ such that for
all $n \geq M$,
%
\begin{equation*}
    d_2(x_n, x) = |x_n - x| < \epsilon
    .
\end{equation*}
%
Since $\cbr{x_n}$ is Cauchy there is an $N \in \N$ such that for all $n,
m \geq N$,
%
\begin{equation*}
    d_2(x_n, x_m) = |x_n - x_m| < \frac{\epsilon}{2 |x + \epsilon|}
    .
\end{equation*}
%
Let $L = \max \cbr{N, M}$. Then, for all $n, m > L$ we have
%
\begin{align*}
    d_2(f(x_n), f(x_m))
        &= |x_n^2 - x_m^2| \\
        &= |x_n - x_m| \, |x_n + x_m| \\
        &= |x_n - x_m| \, |2x + (x_n - x) + (x_m - x)| \\
        &\leq |x_n - x_m| \, |2x + |x_n - x| + |x_m - x|| \\
        &< \frac{\epsilon}{2 |x + \epsilon|} |2x + 2\epsilon| \\
        &= \epsilon
        .
\end{align*}
%
Therefore $\cbr{f(x_n)}$ is Cauchy, and since $(\R, d_2)$ is complete,
the sequence is also convergent.

\end{proof}

\newpage

2. Consider the set of natural numbers $\N$, with the metric functions
defined by
%
\begin{align*}
    d_1(n, m) &\coloneqq \envert{n - m}, \\
    d_2(n, m) &\coloneqq \envert{\frac{1}{n} - \frac{1}{m}}.
\end{align*}
%
(a) Prove that a set is open in $(\N, d_1)$ if and only if it is open in
$(\N, d_2)$.

\begin{proof}

First, note that we can write $d_2$ in terms of $d_1$ as
%
\begin{equation*}
    d_2(n, m) = \frac{d_1(n, m)}{n m}
    .
\end{equation*}
Now, suppose a set $A \subseteq \N$ is open in $(\N, d_1)$. Then,
consider any point $a \in A$. Since $A$ is open, there exists an $r > 0$
such that for all $b \in \N$ satisfying $d_1(a, b) < r$, $b \in A$.

Let $r^\prime = \frac{1}{r}$. Now consider all points $b \in
\N$ satisfying $d_2(a, b) < r^\prime$. For these points we have
%
\begin{align*}
    &d_2(a, b) < r^\prime \\
    &\iff \frac{d_1(a, b)}{a b} < \frac{1}{r} \\
    &\iff d_1(a, b) < r a b  \\
    &\implies d_1(a, b) < r
\end{align*}
%
which further implies that $b \in A$. Therefore $a$ is an interior point
of $A$ in $(\N, d_2)$. Since $a$ was an arbitrary point of $A$, it
follows that $A$ is open in $(\N, d_2)$

Now suppose $A \subseteq \N$ is open in $(\N, d_2)$. Then for every
point $a \in A$ there is an $r > 0$ such that for all $b \in \N$,
$d_2(a, b) < r$ implies $b \in A$.
%
Consider all points $b \in \N$ satisfying $d_1(a, b) < r$. For these
points we have
%
\begin{align*}
    &d_1(a, b) < r \\
    &\iff a b \, d_2(a, b) < r \\
    &\iff d_2(a, b) < r \frac{1}{a b} \\
    &\implies d_2(a, b) < r
\end{align*}
%
which further implies that $b \in A$. Given that $a$ was arbitrary and
that we showed it was an interior point of $A$, it follows that $A$ is
open in $(\N, d_1)$.

\end{proof}

\newpage

(b) Prove that a sequence converges in $(\N, d_1)$ if and only if it
converges in $(\N, d_2)$.

\begin{proof}

Suppose a sequence $\cbr{a_n}$ converges in $(\N, d_1)$ to some $a \in
\N$. Then for any $\epsilon > 0$, there exists an $N \in \N$ such that
for all $n \geq N$,
%
\begin{equation*}
    d_1(a_n, a) < \epsilon
    .
\end{equation*}
%
But then
%
\begin{align*}
    &d_1(a_n, a) < \epsilon \\
    &\iff a_n a \, d_2(a_n, a) < \epsilon \\
    &\implies d_2(a_n, a) < \epsilon
    ,
\end{align*}
%
so $\cbr{a_n}$ also converges in $(\N, d_2)$.

Now suppose $\cbr{a_n}$ is a convergent sequence in $(\N, d_2)$ which
converges to some $a \in \N$. Then for any $\epsilon > 0$ there exists
an $N \in \N$ so that for all $n \geq N$,
%
\begin{equation*}
    d_2(a_n, a) < \frac{\epsilon}{a R}
    ,
\end{equation*}
%
where
%
\begin{equation*}
    R = \frac{1}{\frac{1}{a} - \epsilon} > a_n
    .
\end{equation*}
%
(this is the largest number $R$ satisfying $d_2(R, a) = \epsilon$).

Then we have
%
\begin{align*}
    &d_2(a_n, a) = \envert{\frac{1}{a_n} - \frac{1}{a} }< \epsilon \\
    &\iff \frac{d_1(a_n, a)}{a_n a} < \epsilon \\
    &\implies d_1(a_n, a) < \epsilon
    ,
\end{align*}
%
so $\cbr{a_n}$ also converges in $(\N, d_2)$.

\end{proof}

\newpage

(c) Finally prove that $(\N, d_1)$ is complete, but $(\N, d_2)$ is not.

\begin{proof}

First we will show that $(\N, d_1)$ is complete. Consider any Cauchy
sequence $\cbr{x_n}$ in $(\N, d_1)$. Then for $\epsilon > 0$ with
$\epsilon < 1$ there exists an $N \in \N$ such that for all $m, n \geq
N$,
%
\begin{equation*}
    d_1(x_m, x_n) = |x_m - x_n| < \epsilon
    .
\end{equation*}
%
But since $x_m, x_n \in \N$ the only way to satisfy $|x_m - x_n| < 1$ is
if $x_m = x_n$ for all $m, n \geq N$. Call this constant value $x$. Then
we have that for any $\epsilon > 0$, that for all $n \geq N$ (using the
same $N$ as above), $|x_n - x| = 0 < \epsilon$. Hence the Cauchy
sequence converges, and, since the sequence was arbitrary, it follows
that $(\N, d_1)$ is complete.

Now we will show that $(\N, d_2)$ is not complete. Consider the Cauchy
sequence $s = \cbr{x_n} = \cbr{n}_{n \in \N}$. To show that this
sequence is Cauchy, fix any $\epsilon > 0$. Then let $N = \lceil
\frac{2}{\epsilon} \rceil$. For all $n, m \geq N$, we thus have
%
\begin{align*}
    d_2(x_n, x_m)
        &= \envert{ \frac{1}{n} - \frac{1}{m} } \\
        &< \frac{1}{n} + \frac{1}{m} \\
        &< \frac{1}{N} + \frac{1}{N} \\
        &< \frac{\epsilon}{2} + \frac{\epsilon}{2} \\
        &= \epsilon
        .
\end{align*}
%
However, $s$ fails to converge in $(\N, d_1)$. Suppose for contradiction
that $s$ did converge in $(\N, d_1)$ to some $x \in \N$. Then for any
$\epsilon > 0$ with $\epsilon < 1$ there would be $N \in \N$ such that
for all $n \geq N$
%
\begin{equation*}
    |n - x| < 1
    .
\end{equation*}
%
However, this this is clearly false whenever $n \neq x$, so we have
achieved a contradiction and shown that $s$ does not converge in $(\N,
d_1)$. Following the result from part (b) of this problem, we can
conclude that $s$ does not converge in $(\N, d_2)$, and hence $(\N,
d_2)$ is not complete.

\end{proof}

\newpage

3. Let $\cbr{x_n}$ be a sequence in $\R$ with the standard Euclidean metric.

(a) Suppose that
%
\begin{equation*}
    \envert{x_{n+1} - x_n} < \frac{1}{2^n}, \qquad \forall n \in \N
    .
\end{equation*}
%
Prove that $\cbr{x_n}$ is a Cauchy sequence.

\begin{proof}

Fix any $\epsilon > 0$. Let $N = 1$ if $\epsilon \geq 4$ else let $N =
\lceil - \log_2 \frac{\epsilon}{4} \rceil$. Then, consider all $n, m
\geq N$. If $n = m$ then clearly $|x_n - x_m| = 0$. Suppose that,
without loss of generality, $n \geq m$ and $n - m = k > 0$. Then we have
%
\begin{align*}
    |x_n - x_m|
        &= |(x_n - x_{n - 1}) + (x_{n - 1} - x_{n - 2}) + \cdots + (x_{n - k + 1} - x_m)| \\
        &\leq \sum_{i = 0}^{k - 1} |x_{n - i} - x_{n - i - 1}| \\
        &< \sum_{i = 0}^{k - 1} \frac{1}{2^{n - i - 1}} \\
        &= \frac{2^{n - m} - 1}{2^{n - 1}} \\
        &= 2 (2^{-n} + 2^{-m}) \\
        &< 4 \cdot 2^{-N}
        .
\end{align*}
%
\textbf{Case 1: $\epsilon \geq 4$}. Then we have $N = 1$, and so
%
\begin{equation*}
    |x_n - x_m| < 4 \cdot 2^{-1} = 2 < \epsilon
        .
\end{equation*}

\textbf{Case 2: $\epsilon < 4$}. Then we have $N = \lceil - \log_2
\frac{\epsilon}{4} \rceil$, and so
%
\begin{equation*}
    |x_n - x_m|
        < 4 \cdot 2^{- \del{\lceil - \log_2 \frac{\epsilon}{4} \rceil}}
        < 4 \cdot 2^{- \del{- \log_2 \frac{\epsilon}{4}}}
        < \epsilon
        .
\end{equation*}

In all cases, we have that $|x_n - x_m| < \epsilon$ for $n, m > N$.
Hence $\cbr{x_n}$ is Cauchy.

\end{proof}

\newpage

(b) Is the result in (a) true if we only had
%
\begin{equation*}
    \envert{x_{n+1} - x_n} < \frac{1}{n}, \qquad \forall n \in \N
    ?
\end{equation*}

\begin{proof}

In this case, the sequence is no longer guaranteed to be Cauchy.
Take the example of the sequence $\cbr{x_n}$ whose terms are given by
%
\begin{equation*}
    x_{n + 1} = x_n - \frac{1}{2 n}
    .
\end{equation*}
%
Suppose for contradiction that this sequence  was Cauchy. Then for any
$\epsilon > 0$ there would be an $N \in \N$ such that $n, m > N$ implies
that
%
\begin{equation*}
    |x_n - x_m| < \epsilon
    .
\end{equation*}
%
For a specific example, set $n = N$ and let $m > N$,
Then
%
\begin{align*}
    |x_N - x_m|
        &= |(x_N - x_{N + 1}) + (x_{N + 1} - x_{N + 2}) + \cdots + (x_{N + (m - N - 1)} - x_m)| \\
        &= \envert{ \sum_{i = 0}^{m - N - 1} \del{x_{N + i} - x_{N + i + 1}} } \\
        &= \envert{ \sum_{i = 0}^{m - N - 1} \frac{1}{2 \del{N + i}} } \\
        &= \frac{1}{2} \sum_{i = 0}^{m - N - 1} \frac{1}{N + i} \\
    .
\end{align*}
%
However by setting $m$ sufficiently large, we can make this sum
arbitrarily large (its corresponding infinite series diverges), and
hence we can find an $m > N$ such that $|x_N - x_m| > \epsilon$, which
results in a contradiction. Hence in this case the condition does not
necessarily guarantee a Cauchy sequence.

\end{proof}

\end{document}
